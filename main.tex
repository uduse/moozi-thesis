\documentclass[11pt]{article}
\usepackage[a4paper,
            % inner=10mm,
            % outer=50mm, % = marginparsep + marginparwidth 
                       %   + 5mm (between marginpar and page border)
            % top=20mm,
            % bottom=25mm,
            % marginparsep=5mm,
            % marginparwidth=40mm,
            % showframe
            ]{geometry}
\usepackage{amsfonts}
\usepackage{amsmath}
\usepackage{xcolor}
\usepackage{todonotes}
\usepackage{pythonhighlight}
\usepackage{fancyvrb}
\usepackage{minted}
\usepackage{hyperref}
\hypersetup{linktocpage}

% define command for grey colored text
% \newcommand{\note}[1]{ \textit{\textcolor{gray}{... #1 ...}} }
\newcommand{\note}[1]{\todo[color=yellow!40,bordercolor=none,linecolor=black,inline]{~~ #1}}
\tolerance=1
\emergencystretch=\maxdimen
\hyphenpenalty=10000
\hbadness=10000

\usepackage[font=small,labelfont=bf]{caption}

\newcommand{\includecode}[2]{
\begin{listing}[H]
    \inputminted[frame=single, framesep=5pt, fontsize=\scriptsize]{python}{src/#1.py}
    \caption[]{#2}
    \label{code:#1}
\end{listing}   
}
\renewcommand\listingscaption{Algorithm}

\newcommand{\includeimage}[3][0.7]{
\begin{figure}[H]
    \captionsetup{width=#1\linewidth}
    \centering
    \includegraphics[width=#1\textwidth]{assets/#2.png}
    \caption[]{#3}
    \label{fig:#2}
\end{figure}
}

\newcommand{\argmax}[1]{\operatorname*{argmax}_{#1}}

\usepackage{tikz}
\newcommand*\circled[1]{\tikz[baseline=(char.base)]{
            \node[shape=circle,draw,inner sep=2pt] (char) {#1};}}

% bibilography management
\usepackage[style=numeric]{biblatex}
\addbibresource{main.bib}

\begin{document}

\listoftodos

\tableofcontents

\section{Summary of Notation}
% We adopt a similar notation to \citeauthor{ReinforcementLearningIntroduction_Sutton.Barto_2018} \cite{ReinforcementLearningIntroduction_Sutton.Barto_2018}.
% Capital letters are used for random variables, whereas lower case letters are used for the values of
% random variables and for scalar functions.
% Quantities that are required to be real-valued vectors are written in bold and in lower case (even if random variables).
% Matrices are bold capitals.

\begin{tabular}{| r  l  r |}
    \hline
    \textbf{Symbol} & \textbf{Description}                   & \textbf{Reference}                       \\
    \hline
    $s$             & state                                                                             \\
    $a$             & action                                                                            \\
    $r$             & reward                                                                            \\
    $t$             & timestep                                                                          \\
    $T$             & terminal timestep                                                                 \\
    $o$             & partially observable environment frame                                            \\
    $\gamma$        & discount                                                                          \\
    $\mathbf{x}$    & hidden state                                                                      \\
    $G^N$           & N-step return                                                                     \\
    $\delta$        & TD-error or value diff                 & \ref{sec:replay}                         \\
    $V$             & value function                                                                    \\
    $Q$             & state-action value function                                                       \\
    $\mathcal{A}^e$ & environment action space               & \ref{sec:a_aug}                          \\
    $\mathcal{A}^a$ & agent action space                     & \ref{sec:a_aug}                          \\
    $\mathcal{S}$   & state space                                                                       \\
    $\mathcal{O}$   & observation space                                                                 \\
    $\mathcal{T}_t$ & step sample                            & \ref{sec:targets}                        \\
    $\mathcal{T}$   & trajectory sample                                                                 \\
    $\mathcal{L}$   & loss function                                                                     \\

    \hline
    $h$             & representation function                                                           \\
    $g$             & dynamics function                                                                 \\
    $f$             & prediction function                                                               \\
    $\varrho$       & projection function                                                               \\
    $\mathbf{x}$    & hidden state                                                                      \\
    $v^i$           & value prediction                                                                  \\

    \hline
    $B$             & batch size                                                                        \\
    $H$             & height                                 & \ref{sec:env_bridge}                     \\
    $W$             & width                                  & \ref{sec:env_bridge}                     \\
    $C_e$           & environment channels                   & \ref{sec:env_bridge}                     \\
    $C_h$           & history channels                       & \ref{sec:history_stacking}               \\
    $C_x$           & hidden space channels                  &                \\
    $K$             & number of unrolled steps               & \ref{sec:targets}                        \\
    $L$             & history length                         & \ref{sec:targets}                        \\
    $N$             & bootstrap length for N-step return     & \ref{sec:targets}                        \\
    $A$             & dimension of action                                                               \\
    \hline
    $Z$             & support of the scalar transformation   & \ref{sec:scalar_transform}, \ref{sec:nn} \\
\end{tabular}


% Capital letters are used for random variables, whereas lower case letters are used for the values of random variables and for scalar functions.
% Quantities that are required to be real-valued vectors are written in bold and in lower case (even if random variables).
% Matrices are bold capitals.

% S_0, A_0, R_1, S_1, A_1

\chapter{Introduction} \label{sec:intro}

\textbf{Deep Learning (DL)} is a branch of \textbf{Artificial Intelligence (AI)} that emphasizes the use of neural networks to fit any arbitrary function represented by a dataset.
The training of a neural network is done by computing a loss function from a batch of data, back-propagate gradients with respect to the loss, and updates weights and biases based on the gradients.
Deep learning techniques have been widely adopted in many domains, including computer vision, natural language processing, and robotics.

\textbf{Reinforcement Learning (RL)} is a branch of AI that emphasizes on solving decision making problems through trials and errors with delayed rewards.
RL had most success in the domain of \textbf{game playing}, in which the algorithm is represented as an \textbf{agent} and interacts with the game environments, such as boardgames and Atari games.
An extension to game playing is \textbf{general game playing (GGP)}, whose goal is to design a single agent that can play many different games without having much prior knowledge of the games.

\textbf{Deep Reinforcement Learning (DRL)} is a rising branch that combines DL and RL to solve decision making problems.
In a DRL system, the RL techniques lay out the structure of the algorithm such as the use the \textbf{agent-environment interface}, a value function, a reward signal, e.t.c., while the DL techniques are used approximate specific functions and learn representations.

\textbf{Planning} refers to any computational process that analyzes generated actions and their consequences in an environment.
In the RL terms, planning specifically means the use of a model to improve a policy.
In boardgames where perfect models are accessible, planning with these models yield great performance.
The most significant achievement of planning with a perfect model is AlphaGo beating human champion in Computer Go.
However, how to plan in games where no perfect models available remains a challenging problem to researchers.

A \textbf{distributed system} is a computer system that uses multiple processes with various purposes to complete tasks.
DRL systems for solving large problems are both data and compute intensive.
Utilizing concurrency to increase efficiency and throughput for these DRL systems are sometimes necessary.
Building a distributed system to achieve such concurrency is a common practice in the industry but requires significant engineering effort.

\section{Motivation}
\citeauthor{MasteringAtariGo_Schrittwieser.Antonoglou.ea_2020} developed MuZero, an algorithm that plans with a learned model (\ref{sec:muzero}).
This algorithm achieved the state-of-the-art in playing both Atari games and boardgames.
However, the source code of the algorithm is not publicly available, and the pseudo-code provided with the paper isn't sufficient to reproduce the full algorithm.
Moreover, MuZero requires much more computations than other model-free RL algorithms, and an inefficient implementation will drastically slows down experimentation.
We need an efficient and publicly available implementation of an algorithm that plans with a learned model.
This helps researchers understand how does the algorithm plan with its learned model, and facilitates future research.

\section{Contribution}
In this thesis we present the project \textbf{MooZi}, a system that play games by planning with a learned model.
This project includes:
\begin{itemize}
    \item A collection of environments bridges that connect the system to various common RL environments.
    \item Neural networks that learns representation and can be used for planning.
    \item A MCTS based planner that uses the learned model to perform planning.
    \item A distributed training system that efficiently trains the algorithm.
    \item A thesis with empirical studies and analysis.
\end{itemize}
\chapter{Literature Review} \label{sec:literature}

\section{Planning and Search}
Many AI problems can be reduced to a search problem \cite[p.39]{ArtificialIntelligenceGames_Yannakakis.Togelius_2018}.
Such search problems can be solved by determining the best plan, path, model, function, and so on, based on some metrics of interest.
Therefore, search has played a vital role in AI research since its dawn. The terms \textbf{planning} and \textbf{search} are widely used across different domains.
Here we adopt the definition by \citeauthor{ReinforcementLearningIntroduction_Sutton.Barto_2018} \cite{ReinforcementLearningIntroduction_Sutton.Barto_2018}.

\textbf{Planning} refers to any process by which the agent updates the action selection policy $\pi(a \mid s)$ or the value function $V_\pi(s)$.
We will focus on the case of improving the policy in our discussion.
We view the planning process as an operator $\mathcal{I}$ that takes the policy as input and outputs an improved policy $\mathcal{I}\pi$.

Planning methods can be categorized based on the target state $s$ they aim to improve.
If the method improves the policy for arbitrary states, we call it \textbf{background planning}.
That is, for any timestep $t$ and a set of states $S' \subset \mathcal{S}$:
\begin{align*}
    \pi(a \mid s) \leftarrow \mathcal{I}\pi(a \mid s), ~~ \forall s \in \mathcal{S'}
\end{align*}
Typical background planning methods include \textbf{dynamic programming} and \textbf{Dyna-Q} \cite{ReinforcementLearningIntroduction_Sutton.Barto_2018}.
In the case of dynamic programming, a full sweep of the state space is performed and all states are updated.
In the case of Dyna, a subset of the state space is selected for update.

An other type of planning focuses on improving the policy of the current state $s_t$.
We call this \textbf{decision-time planning}.
That is, for any timestep $t$:
\begin{align*}
    \pi(a \mid s) \leftarrow \mathcal{I}\pi(a \mid s), s = s_t
\end{align*}

Algorithms such as AlphaGo use both types of planning when they use self-play for training.
For decision-time planing, a tree search is performed at the root node and updates the policy of the current state.
For background planning, a neural network uses past experience to train and updates policy for all states.

An early example of the use of search as a planning method is the \textbf{A*} algorithm.
In 1968, \citeauthor{FormalBasisHeuristic_Hart.Nilsson.ea_1968} designed the A* algorithm for finding a shortest path from a start vertex to a target vertex \cite{FormalBasisHeuristic_Hart.Nilsson.ea_1968}.
Although A* works quite well for many problems, especially in early game AI, it falls short in cases where the assumptions of A* do not hold.
For example, A* requires a heuristic, and an optimal solution under stochastic environments.
It is computationally infeasible on large problems.
To address this problem, \citeauthor{RealtimeHeuristicSearch_Korf_1990} framed the problem of \textbf{Real-Time Heuristic Search},
where the agent has to make a decision in each timestep with bounded computation, and developed the \textbf{Real-Time-A*} algorithm as a modified version of A* with bounded computation per step \cite{RealtimeHeuristicSearch_Korf_1990}.
Tree-based search algorithms such as \textbf{MiniMax} and \textbf{Alpha-Beta Pruning} were developed to play and solve two-player games \cite{AnalysisAlphabetaPruning_Knuth.Moore_1975}.
Monte Carlo techniques are designed to handle complex environments.

\section{Monte Carlo Methods}
In 1873, Joseph Jagger observed the bias in roulette wheels at the Monte Carlo Casino.
He studied the bias by recording the results of roulette wheels and won over 2 million francs over several days by betting on the most favorably biased wheel \cite{MonteCarloCasino__2022}.
Therefore, \textbf{Monte Carlo (MC)} methods gained their name as a class of algorithms based on random samplings.

MC methods are used in many domains but in this thesis we will primarily focus on its usage in search.
In a game where terminal states are usually unreachable by the limited search depth, evaluation has to be performed on the leaf nodes that represent intermediate game states.
One way of obtaining an evaluation on a state is by applying a heuristic function.
Heuristic functions used this way are usually hand-crafted by human based on expert knowledge, and hence are prone to human error.
The other way of evaluating the state is to perform a rollout from that state to a terminal state by selecting actions randomly.
This evaluation process is called \textbf{random rollout} or \textbf{Monte Carlo rollout}.

\section{Monte Carlo Tree Search (MCTS)} \label{sec:mcts}

\citeauthor{BanditBasedMonteCarlo_Kocsis.Szepesvari_2006} developed the \textbf{Upper Confidence Bounds applied to Trees (UCT)} method as an extension of the \textbf{Upper Confidence Bound (UCB)} algorithm employed in multi-armed bandit problems \cite{BanditBasedMonteCarlo_Kocsis.Szepesvari_2006}.
Rémi Coulom developed the general idea of \textbf{Monte Carlo Tree Search} that combines Monte Carlo rollouts with tree search \cite{EfficientSelectivityBackup_Coulom_2007} for his Go program CrazyStone.
Shortly afterwards,
\citeauthor{ModificationUCTPatterns_Gelly.Wang.ea_2006} implemented another Go program MoGo that uses the UCT selection formula \cite{ModificationUCTPatterns_Gelly.Wang.ea_2006}.
MCTS was generalized by \citeauthor{MonteCarloTreeSearch_Chaslot.Bakkes.ea_2008} as a framework for game AI \cite{MonteCarloTreeSearch_Chaslot.Bakkes.ea_2008}.
This framework requires less domain knowledge than classic approaches to game AI while giving better results.
% There are four steps in this framework that are iteratively applied to the search tree.
The core idea of this framework is to gradually build the search tree by iteratively applying four steps: \textbf{selection}, \textbf{expansion}, \textbf{evaluation}, and \textbf{backpropagation}.
The search tree built in this way emphasizes more promising moves and game states based on collected statistics in rollouts.
More promising states are visited more often, have more children, have deeper subtrees, and rollout results are aggregated to yield more accurate values. Here we detail the four steps in the MCTS framework by \citeauthor{MonteCarloTreeSearch_Chaslot.Bakkes.ea_2008} (see Figure \ref{fig:mcts}).

\includeimage{mcts}{
    \textbf{The Monte Carlo Tree Search Framework, from \cite{MonteCarloTreeSearch_Chaslot.Bakkes.ea_2008}.}
}{}

\subsection{Selection}
The selection process starts at the root node and repeats until a leaf node in the current tree is reached.
At each level of the tree, a child node is selected based on a selection formula such as UCT or PUCT.
A selection formula usually has two parts: the exploitation part based on the evaluation function $E$, and the exploration bonus function $B$.
For actions $(s, a), a \in \mathcal{A}$ of a parent state $s$ , the selection $I(s)$ is defined as
\begin{align}
    I(s) = \argmax{a \in \mathcal{A}}{\left[ E(s, a) + B(s, a) \right]}
    \label{eq:mcts_selection}
\end{align}

The evaluation function $E$ can be based on the value of the child, the accumulated reward of the child, or the prior selection probability based on the policy $\pi(a \mid s)$.
The exploration bonus function $B$ is usually based on the visit count of the child and the parent.
The more visits a child has, the smaller the exploration bonus becomes.
For example, the UCT algorithm uses
\begin{align*}
    % I(s)     & = \argmax{a \in \mathcal{A}}{ \left( E(s, a) + B(s, a) \right)}  \\
    E(s, a)  & = \frac{V(s)}{N(s, a)}  \\
    B(s, a)  & = \sqrt{\frac{2 * \log(\sum_{b \in \mathcal{A}}N(s, b))}{N(s, a)}}
\end{align*}
where $V(s)$ is the value of the node, and $N(s, a)$ is the visit count of the edge.
This \citeauthor{ModificationUCTPatterns_Gelly.Wang.ea_2006} used this selection rule in their implementation of MoGo,
the first computer Go program that uses UCT \cite{ModificationUCTPatterns_Gelly.Wang.ea_2006}.
\citeauthor{MultiarmedBanditsEpisode_Rosin_2011} developed the PUCB and the PUCT algorithm that utilize a predictor $P(s, a)$ that estimates the prior probability of the action $a$ being selected from state $s$ and later being used in AlphaGo (\ref{sec:puct}, \cite{MultiarmedBanditsEpisode_Rosin_2011}).

\subsection{Expansion}
The selected leaf node is expanded by adding one or more children.
Each child represents a successor game state reached by playing the associated legal move.

\subsection{Evaluation}
The expanded node is evaluated, either by playing a game with a rollout policy, or by using an evaluation function, or by using a blend of both approaches.
Many MCTS algorithms use a randomized policy as the rollout policy and the game result as the evaluation.
Early work on evaluation functions focused on hand-crafted or machine learned heuristic functions based on expert knowledge.
Recently, evaluation functions use deep neural networks specifically trained for the problems (\ref{sec:alpha_go}).

\subsection{Backpropagation}
After the expanded nodes are evaluated, the nodes on the path from the expanded nodes back to the root are updated.
The statistics updated usually include visit count, estimated value and accumulated reward of the nodes.

\subsection{MCTS Iteration and Move Selection}
The four MCTS steps are repeated until the budget runs out.
The budget is usually a limited number of simulations or a period of time.
After the search, the agent acts by selecting the action associated with the most promising child of the root node.
This could be the most visited child, the child with the greatest value, or the child with the greatest lower confidence bound value \cite{FreshMaxLcb_RoyJonathan_2019,AcceleratingSelfPlayLearning_Wu_2020}.

\section{AlphaGo} \label{sec:alpha_go}
In \citeyear{MasteringGameGo_Silver.Schrittwieser.ea_2017},
\citeauthor{MasteringGameGo_Silver.Schrittwieser.ea_2017} developed \textbf{AlphaGo},
the first Go program that beat a human Go champion on even terms \cite{MasteringGameGo_Silver.Schrittwieser.ea_2017}.
AlphaGo was trained with a machine learning pipeline with multiple stages.
For the first stage of training, a supervised learning policy (or SL policy) is trained to predict expert moves using a neural network.
This SL policy $p$ is parametrized by weights $\sigma$, denoted $p_{\sigma}$.
The input of the policy network is a representation of the board state, denoted $s$.
The network outputs a probability distribution over all legal moves $a$ through the last softmax layer.
During the training of the network, randomly sampled expert moves are used as training targets.
The weights $\sigma$ are then updated through gradient ascent to maximize the probability of matching human expert move:
$$
    \Delta \sigma \propto \frac{\partial \log p_{\sigma}(a \mid s)}{\partial \sigma}
$$
For the second stage of training, the supervised policy $p_{\sigma}$ is used as the starting point for training with reinforcement learning.
This reinforcement learning trained policy (or RL policy) is parametrized by weights $\rho$ and is initialized $p_{\rho} = p_{\sigma}$.
Training data is generated in form of self-play games using $p_{\rho}$ as the rollout policy.
For each game, the game outcome $z_t = \pm r(s_T)$, where $s_T$ is the terminal state, $z_T = +1$ for winning, $z_T = -1$ for losing from the perspective of the current player.
Weights $\rho$ are updated using gradient ascent to maximize the expected outcome using the update formula:
$$
    \Delta \rho \propto \frac{\partial \log p_{\rho}\left(a_{t} \mid s_{t}\right)}{\partial \rho} z_{t}
$$
Finally, a value function is trained to evaluate board positions.
This value function is modeled with a neural network with weights $\theta$, denoted $V_{\theta}$.
Given a state $s$, $V_{\theta}(s)$ predicts the outcome of the game if both players act according to the policy $p_{\rho}$.
This neural network is trained with stochastic gradient descent to minimize the mean squared error (MSE) between the predicted value $V_{\theta}(s)$ and the outcome $z$.
$$
    \Delta \theta \propto \frac{\partial V_{\theta}(s)}{\partial \theta}\left(z-V_{\theta}(s)\right)
$$

AlphaGo combines the policy network $p_{\rho}$ and the value network $V_{\theta}$ with MCTS for acting.
AlphaGo uses a MCTS variant called PUCT similar to that described in \ref{sec:mcts}.
In the search tree, each edge $(s, a)$ stores an action value $Q(s, a)$, a visit count $N(s, a)$, and a prior probability $P(s, a)$.
At each time step, the search starts at the root node and simulates until the budget runs out.
In the select phase of each simulation, an action is selected for each traversed node using the same base formula in Equation \ref{eq:mcts_selection}.
In AlphaGo, the exploitation score of the selection formula is the estimated average value of the next state after taking the actions, namely $Q(s, a)$.
In AlphaGo's PUCT formula, The exploration bonus of edge $(s, a)$ is based on the prior probability $P$ and decays as its visit count $N$ grows.
As before, the action taken at time $t$ maximizes the sum of the exploitation score and the exploration bonus
\begin{align*}
    I(s)     & = \argmax{a \in \mathcal{A}}\left[ E(s, a)+B(s, a) \right]  \\
    E(s, a)  & = Q\left(s, a\right)  \\
    B(s, a)  & \propto \frac{P(s, a)}{1 + N(s, a)}  \\
\end{align*}

AlphaGo evaluates a leaf node state $s_L$ by blending both the value network estimation $V_\theta(s_L)$ and the game result $z_L$ obtained by the rollout policy $p_\pi$
The mixing parameter $\lambda \in [0, 1]$ is used to balance these two types of evaluations into the final evaluation $V(s_L)$
$$
    V\left(s_{L}\right)=(1-\lambda) V_{\theta}\left(s_{L}\right)+\lambda z_{L}
$$

\section{AlphaGo Zero}
\textbf{AlphaGo Zero} is a successor of AlphaGo that beat AlphaGo by 100-0 in 100 games \cite{MasteringGameGo_Silver.Schrittwieser.ea_2017}.
In contrast, AlphaGo Zero learns to play Go from \textit{tabula rasa}.
This means it learns solely by reinforcement learning from self-play, starting from random play, without supervision from human expert data.

Central to AlphaGo Zero is a deep neural network $f_\theta$ with parameters $\theta$.
Given a state $s$ as an input, the network outputs both move probabilities $\pmb{p}$ and value estimation $v$
\begin{align*}
    (\mathbf{p}, v) = f_\theta(s)
\end{align*}
To generate self-play games $s_1, ..., s_T$, MCTS is performed at each state $s$ using the latest neural network $f_\theta$.
To select a move for a parent node $p$ in the search tree, a variant of the PUCT algorithm is used:
\begin{align*}
    I(s)     & = \operatorname{argmax}_{a \in \mathcal{A}} \left( E(s, a) + B(s, a) \right)  \\
    E(s, a)  & = Q(s, a)  \\
    B(s, a)  & \propto P(s, a) \frac{\sqrt{\sum_{b \in \mathcal{A}}{N(s, b)}}}{1+N(s, a)}
\end{align*} \label{sec:puct}.

Self-play games are processed into training targets to update the network parameters $\theta$ through gradient descent on the loss function $l$
\begin{equation*}
    \mathcal{L}(\theta) = (z-v)^{2} - \pmb{\pi}^{\mathrm{T}} \log \pmb{p}+c\|\theta\|^{2}
\end{equation*}
Here $(z-v)^2$ is the mean squared error of the prediction value,
$-\pmb{\pi}^{\mathrm{T}} \log \pmb{p}$ is the cross-entropy loss of the move probabilities,
and $c\|\theta\|^2$ is a $L_2$ weight regularization.
Many other components of this system are similar to those in AlphaGo.

\section{AlphaZero}
\textbf{AlphaZero} reduces game specific knowledge of AlphaGo Zero even further so that the same algorithm can be also applied to Shogi and chess
\cite{MasteringChessShogi_Silver.Hubert.ea_2017}.
One generalization is that in AlphaZero the game result is no longer either winning or losing ($z \in \{ -1, +1 \}$), but can also be a draw ($z \in \{-1, 0, +1 \}$).

\section{\textit{MuZero}} \label{sec:muzero}
In \citeyear{MasteringAtariGo_Schrittwieser.Antonoglou.ea_2020},
\citeauthor{MasteringAtariGo_Schrittwieser.Antonoglou.ea_2020} developed
\textbf{\textit{MuZero}}, an even more algorithm that learns to play Atari, Go, chess and Shogi at superhuman level.
Compared to the AlphaGo and AlphaZero,
MuZero has no access to a perfect model of the game.
MuZero plans with a neural network that learns the game dynamics through experience.
Therefore, MuZero can be applied to games where either the perfect model is not known or is infeasible to compute with.

MuZero defines three main functions.
The \textbf{representation function} $h$ encodes a history of observations $o_1, o_2, \dots, o_t$ and actions $a_1, a_2, \dots, a_{t - 1}$ into a hidden state $\mathbf{x}_t^0$.
This hidden state is learned, and is the main conceptual change from AlphaZero.
The \textbf{dynamics function} $g$ implements action execution in the representation.
Given a hidden state $\mathbf{x}^k$ and action $a^k$, produces an immediate reward $r^k$ and the next hidden state $\mathbf{x}^{k+1}$.
The \textbf{prediction function} $f$ corresponds to the one network in AlphaZero.
Given a hidden state $\mathbf{x}^k$, it produces a probability distribution $p^k$ of actions and a value $v^k$ associated to that hidden state.
Three functions $f, g, h$ are approximated jointly in a neural network with weights $\theta$
\begin{align}
    \mathbf{x}^0_t                     & = h_{\theta}(o_1, \dots, o_t, a_1, \dots, a_{t - 1}) \label{eq:muzero_h}  \\
    (\mathbf{x}^{k+1}, \hat{r}^{k+1})  & = g_{\theta}(\mathbf{x}^k, a^k)  \label{eq:muzero_g}  \\
    (v^k, \pmb{p}^k)                   & = f_{\theta}(\mathbf{x}^k) \label{eq:muzero_f}
\end{align}
The superscripts of $\mathbf{x}, a, v$ denote the depth of such values in the search tree, and depth $0$ is at the search tree's root.
Equivalently, the superscripts also mean the number of recurrent inferences (through the dynamics function $g$) the algorithm performs to obtain that value.

MuZero plans with a search method based on the MCTS framework (discussed in \ref{sec:mcts}).
Due to the lack of access to a perfect model, MuZero's MCTS differs from a standard one in numerous ways.
The nodes are no longer perfect representations of the board states.
Instead, each node is associated with a hidden state $\mathbf{x}$ as a learned representation of the board state.
The transition is no longer made by the perfect model but by the dynamics function $g$.
Moreover, since the dynamics function also predicts a reward, edges created through inferencing with the dynamics function also contribute to the $Q$ value estimation.
% That is, traditionally we have $Q(s, a) = \mathit{E} \left[ V(S_{t + 1}) \mid S_t = s, A_t = a \right]$.

To act in the environment, MuZero plans following the MCTS framework described in section \ref{sec:mcts}.
At each timestep $t$, $\mathbf{x}^0_t$ is created using (\ref{eq:muzero_h}).
% To select an action following the MCTS selection template equation , 
A variant of PUCT is used to select an action during the search
\begin{align*}
    I(s)     & = \argmax{a \in \mathcal{A}} \left( E(s, a) + B(s, a) \right)  \\
    E(s, a)  & = Q(s, a)  \\
    B(s, a)  & \propto P(s, a) \frac{\sqrt{\sum_{b \in \mathcal{A}}N(s, b)}}{1+N(s, a)}\left[c_{1}+\log \left(\frac{\sum_{b \in \mathcal{A}}N(s, b)+c_{2}+1}{c_{2}}\right)\right] ~~~~.
\end{align*}
Where $c_1$ and $c_2$ are two constants that adjust the exploration bonus.
The selected edge $(\mathbf{x}^k, a^k)$ at depth $k$ is expanded using (\ref{eq:muzero_g}) and evaluated using (\ref{eq:muzero_f}).
At the end of the simulation, the statistics of the nodes along the search path are updated.
We denote the updated prior action probabilities $\mathbf{p}^*$, and the updated value estimation $v^*$.
Notice since the transitions of the nodes are approximated by the neural network, the search is performed over hypothetical trajectories without using a perfect model.
Finally, the action $a^0$ of the most visited edge $(\mathbf{x}^0, a^0)$ of the root node is selected as the action to take in the environment.

Experience generated is stored in a replay buffer and processed into training targets.
The three functions of the model are trained jointly using the loss function
\begin{equation}
    \mathcal{L}_{t}(\theta)=
    \underbrace{\sum_{k=0}^{K} \mathcal{L}^{p}\left(p^*_{t+k}, p_{t}^{k}\right)}_{\circled{1}}
    +
    \underbrace{\sum_{k=0}^{K} \mathcal{L}^{v}\left(z_{t+k}, v^*_{t}\right)}_{\circled{2}}
    +
    \underbrace{\sum_{k=1}^{K} \mathcal{L}^{\mathrm{r}}\left(r_{t+k}, \hat{r}^{k}\right)}_{\circled{3}}
    +
    \underbrace{c\|\theta\|^{2}}_{\circled{4}}
\end{equation}
where $K$ is the rollout depth, \circled{1} is the loss of the predicted prior move probabilities and move probabilities improved by the search, \circled{2} is the loss of the predicted value and experienced N-step return,
\circled{3} is the loss of the predicted reward and the experienced reward, and finally \circled{4} is the $L_2$ regularization.

\subsection{MuZero Reanalyze} \label{sec:muzero_reanalyze}
\citeauthor{MasteringAtariGo_Schrittwieser.Antonoglou.ea_2020} also developed \textbf{MuZero Reanalyze}, a sample efficient variant of MuZero \cite{MasteringAtariGo_Schrittwieser.Antonoglou.ea_2020}.
This method generates training targets in addition to those generated through game play through re-executing search on old games using the latest parameters.
\textbf{MuZero Unplugged} and \textbf{Efficient Zero} also use similar mechanisms to generate new data by updating search statistics of old data \cite{OnlineOfflineReinforcement_Schrittwieser.Hubert.ea_2021}.
In Efficient Zero, experiments are ran with a reanalyze ratio of 0.99, which means only $1\%$ of the training data are generated through interacting with the environment, and the other 99\% are generated by re-running search on old trajectories.
In our project, we also implement a reanalyze worker to perform this task (see Section \ref{sec:reanalyze}, \ref{sec:re_w}).

\section{Atari Games Playing}
\subsection{Atari Learning Environment} \label{sec:ale}
\textbf{The Atari 2600} gaming console was developed by \textit{Atari, Inc.} and was released in 1977.
Over 30 million copies of the console sold over its 15 years on the market \cite{Atari2600__2022}.
The most popular game, PacMan, was sold over 8 million copies and was the all-time best-selling video game back then.
\textbf{Stella} is a multi-platform Atari 2600 emulator released under the GNU General Public License (GPL) \cite{StellaMultiPlatformAtari__}.
Stella was ported to popular operating systems such as Linux, MacOS, and Windows, providing Atari 2600 experiences to users without physical copies of the equipment.
In \citeyear{ArcadeLearningEnvironment_Bellemare.Naddaf.ea_2013}, \citeauthor{ArcadeLearningEnvironment_Bellemare.Naddaf.ea_2013} introduced the \textbf{Arcade Learning Environment (ALE)} and the library has been publicly available since \cite{ArcadeLearningEnvironment_Bellemare.Naddaf.ea_2013}.
ALE provides interfaces of over a hundred of Atari game environments using Stella as the backend.
Each ALE environment has specifications on its visual representation, action space, and reward signals.
ALE environments are suitable for controlled machine learning research,
because data are well-represented and evaluation metrics are clearly defined.
Moreover, ALE environments are diverse in their characteristics: while some environments require more mechanical mastery of the agent, others require more long-term planning.
This makes solving multiple ALE environments using the same algorithm a good general game playing problem (\ref{sec:intro}).

\subsection{Deep Q-Networks} \label{sec:dqn}
\citeauthor{PlayingAtariDeep_Mnih.Kavukcuoglu.ea_2013} pioneered the study of using deep neural networks to learn in ALE environments \cite{PlayingAtariDeep_Mnih.Kavukcuoglu.ea_2013}.
They developed the algorithm \textbf{Deep Q-Networks (DQN)} that learned to play seven of the Atari games and reached human-level performance.
The DQN agent has a neural network that approximates the $Q$ function, parametrized by weights $\theta$, denoted $Q_\theta$.
Experiences are generated through interacting with the environment by taking the action that maximizes the immediate $Q$ value
\begin{align*}
    \pi(a_t \mid (o_{t - L + 1}, \dots, o_t)) = \argmax{a}{Q_{\theta}(o_{t - L + 1}, \dots, o_t, a)}
\end{align*}
where $L$ is the length of history, and $o_t$ is the ``frame'', a partial observation of the game state at timestep $t$ (also see \ref{sec:history_stacking}).
Generated experience is stored in an experience replay buffer implemented as a FIFO queue.
For each training step, a batch of uniformly sampled experience is drawn from the experience replay, and the loss is computed using
\begin{equation*}
    \mathcal{L}(\theta) \propto \mathbb{E}_\pi\left[r + \gamma \max _{a'} Q_{\theta'}(s', a') - Q_{\theta}(s, a) \right] ~~~~ .
\end{equation*}
The network parameters $\theta'$ are updated less frequently than $\theta$.

\subsection{Double Q Learning}
\citeauthor{DoubleQlearning_Hasselt_2010} analyzed the overestimation problem of Q values in Q-learning and developed \textbf{double Q learning}, where a double Q update replaces the traditional Q update \cite{DoubleQlearning_Hasselt_2010}.
Double Q learning reduces the overestimation problem by introducing an additional Q estimator's and updating two estimator using each other
\begin{align*}
    Q^{A}(s, a) \leftarrow Q^{A}(s, a)+ \alpha \left(r+\gamma Q^{B}\left(s', \argmax{a'}{Q^A(s', a')}\right)-Q^{A}(s, a)\right)  \\
    Q^{B}(s, a) \leftarrow Q^{B}(s, a)+ \alpha \left(r+\gamma Q^{A}\left(s', \argmax{a'}{Q^B(s', a')}\right)-Q^{B}(s, a)\right)
\end{align*}
where $Q^A$ and $Q^B$ are two different Q estimators updated alternately.
\citeauthor{DeepReinforcementLearning_Hasselt.Guez.ea_2016} applied the double Q learning in DQN \cite{DeepReinforcementLearning_Hasselt.Guez.ea_2016}.
Similar to the double Q update above, a double Q update for neural networks is formulated as
\begin{align*}
    \mathcal{L}(\theta^A)  & \propto \mathbb{E}_\pi \left[ r + \gamma Q_{\theta^B}\left(s', \argmax{a'}{Q_{\theta^A}(s', a')} \right) - Q_{\theta^A}(s, a) \right]  \\
    \mathcal{L}(\theta^B)  & \propto \mathbb{E}_\pi \left[ r + \gamma Q_{\theta^A}\left(s', \argmax{a'}{Q_{\theta^B}(s', a')} \right) - Q_{\theta^B}(s, a) \right] ~~ .
\end{align*}
Here $Q_{\theta^A}$ and $Q_{\theta^B}$ are two sets of parameters of the same neural network architecture.

\subsection{Experience Replay}
\citeauthor{PrioritizedExperienceReplay_Schaul.Quan.ea_2016} studied the role of a experience replay in DQN and developed the \textbf{prioritized experience replay} method \cite{PrioritizedExperienceReplay_Schaul.Quan.ea_2016}.
In the original work of DQN, all samples were drawn from the experience replay uniformly.
In prioritized experience replay, however, samples are drawn according to a distribution based on their calculated priority
\begin{align*}
    P(i)=\frac{p_{i}^{\alpha}}{\sum_{k} p_{k}^{\alpha}}
\end{align*}
where $P(i)$ is the probability of the $i$-th sample being drawn, $\alpha$ is a constant, and $p_i$ is the priority of the sample.
\citeauthor{PrioritizedExperienceReplay_Schaul.Quan.ea_2016} developed two approaches to compute priorities of samples.
In \textbf{proportional sampling}, the priority $p$ of sample $i$ is calculated by
\begin{align*}
    p_i = \left|\delta_{i}\right|+\epsilon
\end{align*}
where $\delta_{i}$ is the temporal-difference error of the sample, and $\epsilon$ is a small constant to give all samples a non-zero probability to be drawn.
In \textbf{rank-based sampling}, the same temporal differences are calculated, but the final priority is computed based on the rank of the error,
\begin{align*}
    \text{score($i$)}  & = \left|\delta_{i}\right|+\epsilon  \\
    p_{i}              & = \frac{1}{\operatorname{rank}(\operatorname{score}(i))}
\end{align*}
\citeauthor{DistributedPrioritizedExperience_Horgan.Quan.ea_2022} followed up by implementing a distributed version of prioritized experience replay \cite{DistributedPrioritizedExperience_Horgan.Quan.ea_2022}.
\citeauthor{RecurrentExperienceReplay_Kapturowski.Ostrovski.ea_2022} investigated the challenges of using experience replays for RNN-based agents and developed \textbf{Recurrent Replay Distributed DQN} \cite{RecurrentExperienceReplay_Kapturowski.Ostrovski.ea_2022}.

\subsection{Network Architectures}
\citeauthor{DuelingNetworkArchitectures_Wang.Schaul.ea_2016} studied an alternative neural network architecture for ALE learning \cite{DuelingNetworkArchitectures_Wang.Schaul.ea_2016}.
\textbf{Dueling Q-network} retains the input and output specifications of the Q-network used in DQN and structurally represented the learning of the advantage function $A(s, a)$ defined as
\begin{align*}
    A(s, a) = Q(s, a) - V(s)
\end{align*}
The Q-network has three parts:
$\theta$, the shared trunk of the network; $\varLambda$, the advantage head; and $\varUpsilon$, the value head.
The network approximates the value function internally through the shared trunk and the value head, denoted $V_{\theta, \varUpsilon}$, and the advantage function, denoted $A_{\theta, \varLambda}$.
The values computed by the two heads are combined to form the Q-value as follows
\begin{align*}
    Q_{\theta, \varUpsilon, \varLambda}(s, a)
    = V_{\theta, \varUpsilon}(s)
    + \left( A_{\theta, \varLambda}(s, a)
    - \frac{1}{| \mathcal{A} | } \sum_{a'}A_{\theta, \varLambda}(s, a') \right)
\end{align*}
Similar to DQN, the dueling Q-network is trained through fitting to empirical data generated by interacting with the environment.
Experiments show that this architecture encourages the network to learn to differentiate between the values of states and the values of state-action pairs, and leads to better performance of the agent.

\subsection{Scalar Transformation} \label{sec:scalar_transform}
\citeauthor{ObserveLookFurther_Pohlen.Piot.ea_2018} introduced enhancements to achieve more stable training in Atari games \cite{ObserveLookFurther_Pohlen.Piot.ea_2018}.
We focus on discussing the \textbf{transformed Bellman Operator} since both MuZero and MooZi use it.
For different Atari games, reward signals can vary drastically both in density and scale.
This leads to high variance in training targets during training of the algorithms, causing algorithms to have difficulty converging.
In DQN, rewards are clipped the reward signal to a range of $[-1, 1]$ to reduce such variance \cite{PlayingAtariDeep_Mnih.Kavukcuoglu.ea_2013}.
However, this clipping discards the scale of rewards and consequently changes the set of optimal policies.
The transformed Bellman Operator was developed to address this problem.
The $Q$ update of the new operator is as follows
\begin{align*}
    % Q^{A}(s, a) \leftarrow Q^{A}(s, a)+ \alpha \left(r+\gamma Q^{B}\left(s', \argmax{a'}{Q^A(s', a')}\right)-Q^{A}(s, a)\right)  \\
    Q(s, a) \leftarrow Q(s, a) + \alpha \phi \left(r +\gamma \max _{a' \in \mathcal{A}} \phi^{-1}\left(Q\left(s', a'\right)\right)\right)
\end{align*}
where $\phi$ is an invertible transformation that contracts.
One example of such a transformation is
\begin{align*}
    \phi(x)       & = \operatorname{sign}(x)\left(\sqrt{|x|+1}-1\right)+\varepsilon x  \\
    \phi^{-1}(x)  & = \operatorname{sign}(x)\left(\left(\frac{\sqrt{1+4 \varepsilon(|x|+1+\varepsilon)}-1}{2 \varepsilon}\right)^{2}-1\right)
\end{align*}
Both \textit{MuZero} and \moozi use this specific $\phi$ definition for both value transformations and reward transformations (\ref{sec:nn}).

% \subsection{Efficiency Problems with ALE} \label{sec:eff_ale}
% % ALE environments are slower transition than environments of board games.
% For algorithms like DQN, more time are spent on stepping the environments than computing policies.
% One key reason is that policies based on neural networks can be computed using specialized hardwares such as GPUs and TPUs.
% ALE environments are CPU-only and linearly increasing number of environments always also linearly increase their memory and CPU cycle consumptions.
% Data hungry algorithms such as MuZero were trained with 20 billion environment frames, 1 million of training steps, and lasted more than 12 hours.
% It is difficult for researchers with limited computation resources to produce such work, let alone extending it.

\subsection{MinAtar} \label{sec:min_atar}
\textbf{MinAtar}, developed by \citeauthor{MinAtarAtariInspiredTestbed_Young.Tian_2019}, is an open-source project that offers RL environments inspired by ALE \cite{MinAtarAtariInspiredTestbed_Young.Tian_2019}.
\textit{MinAtar} offers five environments that pose similar challenges to ALE environments: learning representation from raw pixels, and learning behaviors that associate actions and delayed rewards.
\textit{MinAtar} environments are implemented in pure Python, have simpler environment dynamics, and are visually less rich than ALE environments.
This makes \textit{MinAtar} perfect test environment for university research.

\subsection{Consistency Loss}
One interesting characteristic of Atari-like games is that the environment frames are usually temporally consistent.
For example, given the position of the player avatar for the last few frames, it is not difficult for a human to guess the position of the avatar in the next frame.
To take advantage of this property, one common approach is to enforce temporal consistency in the loss function.
\Citeauthor{VisualizingMuZeroModels_deVries.Voskuil.ea_2021} visualized the latent space of a learned model of MuZero in a 3D space, in which a hidden state is a point in the space \cite{VisualizingMuZeroModels_deVries.Voskuil.ea_2021}.
As MuZero applies recurrent inferences to a hidden state, the transitions can be traced as a 1-D path in the 3D space.
The consistency loss they developed creates a smoother path in the 3D space and improves performance.
\Citeauthor{MasteringAtariGames_Ye.Liu.ea_2021} developed a project-then-predict structure similar to a Siamese network to enforce consistency \cite{MasteringAtariGames_Ye.Liu.ea_2021,SiameseNeuralNetworks_Koch.Zemel.ea_2015}.

\section{Deep Reinforcement Learning Systems} \label{sec:drl_systems}
Deep reinforcement learning systems involve irregular computation patterns and complicated hardware interactions between CPUs and AI accelerators.
Designing such systems efficiently a great challenge.
Decisions the designer has to make include but are not limited to (1) Where and how to generate experience? (2) Where and how to store generated experience? (3) Where to store the model and copies of it? (4) Where is the gradient computation carried out? (4) How to orchestrate processes for stable training?
Here we briefly review popular deep reinforcement learning system designs that utilize parallelization to achieve faster and more efficient training.

\subsection{\Citeauthor{AsynchronousMethodsDeep_Mnih.Badia.ea_2016}'s Asynchronous Methods Framework}
\citeauthor{AsynchronousMethodsDeep_Mnih.Badia.ea_2016} developed asynchronous variants for four popular RL algorithms with a parallelization structure uses actor-learner processes \cite{AsynchronousMethodsDeep_Mnih.Badia.ea_2016}.
Each actor-learner process holds a local copy of the model, generates experience locally using the model, and accumulates gradients locally.
Once in a while, all local gradients are aggregated to update the global model.
Delaying and aggregating updates to neural network parameters reduces gradient variance among processes and achieves a more stable learning.
Among the asynchronous algorithm variants, \textbf{Asynchronous Advantage Actor Critic (A3C)} had the best performance and achieved the state-of-the-art at the time using only half the training time.

\subsection{The IMPALA Architecture} \label{sec:impala}
\citeauthor{IMPALAScalableDistributed_Espeholt.Soyer.ea_2018} developed \textbf{IMPALA}, a scalable distributed deep reinforcement learning agent \cite{IMPALAScalableDistributed_Espeholt.Soyer.ea_2018}.
IMPALA deploys two types of computation workers: \textit{actor} and \textit{learner}.
A actor holds a copy of the neural network parameters and the environment.
It performs model inferences locally to interact with its environments and generates experiences.
Generated experiences are saved in a local storage and subsequently pushed into the learner's local storage.
The learner holds the master copy of the neural network parameters.
Once the learner receives enough experiences from the actors, it samples experiences from its local queue and performs batched forward pass and back-propagation steps using its model.
Figure \ref{fig:impala} shows two variants of this structure.
\includeimage[0.8]{impala}{
    \textbf{IMPALA Architecture, from \cite{IMPALAScalableDistributed_Espeholt.Soyer.ea_2018}.}.
}{
    \textit{Left}: a single learner computes all gradients;
    \textit{Right}: multiple worker learners compute gradients and one master learner collects and aggregates gradients.
}

\subsection{The SEED Architecture}
\citeauthor{SEEDRLScalable_Espeholt.Marinier.ea_2020} developed the \textbf{Scalable, Efficient Deep-RL (SEED)} architecture to effectively utilize accelerators using a centralized inference server \cite{SEEDRLScalable_Espeholt.Marinier.ea_2020}.
Similar to IMPALA, SEED also uses two main types of workers: actors and learners.
However, in SEED, actors do not hold copies of the model.
Instead, SEED actors interact with their environments through querying the learner.
The learner not only computes gradients and stores trajectories as in IMPALA, but also has a batching layer that batches actor queries and efficiently performs batched inference with the model.
Since actors no longer need to pull neural network parameters from the learner, the IO overhead from serializing and messaging parameters is eliminated.
Moreover, since the learner batches queries from all actors, the IO overhead from moving inputs and outputs to accelerators (GPUs or TPUs) is also reduced, increasing the overall inferencing throughput.
One downside of the SEED architecture is that actors have to wait for a response from the learner to take an action, and thus have a higher latency for taking a step.
Figure \ref{fig:seed} illustrates a distributed SEED agent.
\includeimage[0.8]{seed}{
    \textbf{The SEED Architecture, from \cite{SEEDRLScalable_Espeholt.Marinier.ea_2020}}.
}{
    All inferences are computed on the learner and actors act through querying the learner.
}

\subsection{The Acme Framework}
\citeauthor{AcmeResearchFramework_Hoffman.Shahriari.ea_2020} developed the \textbf{Acme} research framework \cite{AcmeResearchFramework_Hoffman.Shahriari.ea_2020}.
Acme similar to IMPALA:
processes that interact with the environment are actors,
and processes that collect experience and update gradients are learners.
Additionally, Acme has a \textit{dataset} component, which is synonymous to the replay buffer used in DQN.
This component uses \textbf{Reverb}, a high-performance library developed by \citeauthor{ReverbFrameworkExperience_Cassirer.Barth-Maron.ea_2021} for storing and sampling collected experiences \cite{ReverbFrameworkExperience_Cassirer.Barth-Maron.ea_2021}.
Figure \ref{fig:acme} illustrates a distributed asynchronous agent in Acme.
% Our project has a similar structure to Acme framework
\includeimage[0.5]{acme}{
    \textbf{Example of a distributed asynchronous agent with Acme, from \cite{AcmeResearchFramework_Hoffman.Shahriari.ea_2020}.}
}{}

\subsection{Ray and RLlib} \label{sec:ray}
\citeauthor{RayDistributedFramework_Moritz.Nishihara.ea_2018} designed and implemented \textbf{Ray}, a framework for scalable distributed computing \cite{RayDistributedFramework_Moritz.Nishihara.ea_2018}.
Ray enables both task-level and actor-level parallelization through a unified interface.
\textbf{Ray Core} was designed with AI applications in mind and has powerful primitives for building distributed AI systems.
For example, Ray uses shared memory to store inputs and outputs of tasks, allowing zero-copy data sharing among tasks.
This is useful for DRL systems in which generated experiences are stored and sampled in a separate process.
\citeauthor{RLlibAbstractionsDistributed_Liang.Liaw.ea_2018} developed \textbf{RLlib}, an industrial-grade deep reinforcement learning library.
RLlib is built on top of Ray Core and provides abstractions for a broad range of DRL systems could make use of.
Figure \ref{fig:rllib} illustrates RLlib's abstraction layers.
As of the writing of this thesis, RLlib implemented 24 popular DRL algorithms using its abstractions.
One major difference between RLlib agents and other DRL agents is that RLlib deploys a hierarchical control over the worker processes.
Our project uses Ray Core to implement its worker processes and deploys a hierarchical control paradigm similar to RLlib (see \ref{sec:method}).
\includeimage[0.5]{rllib}{
    \textbf{RLlib Abstraction Layers, from \cite{RLlibAbstractionsDistributed_Liang.Liaw.ea_2018}.}
}{}

\subsection{JAX and Podracer Architecture} \label{sec:jax_and_podracer}
\citeauthor{CompilingMachineLearning_Frostig.Johnson.ea_2018} designed \textbf{JAX}, a just-in-time (JIT) compiler that compiles computations expressed in Python code into high-performance accelerators code \cite{CompilingMachineLearning_Frostig.Johnson.ea_2018}.
JAX is compatible with \textbf{Autograd}, so computation procedures expressed and compiled with JAX can be automatically differentiated.
JAX also supports control flow, allowing more sophisticated logic to be expressed while taking advantage of accelerators.
Our project uses JAX for both neural networks and search.
As a result, we are able to compile the entire policy in rollout workers, including history stacking, planning, and neural networks inferencing, into a single optimized program that can be hardware-accelerated.
\citeauthor{PodracerArchitecturesScalable_Hessel.Kroiss.ea_2021} designed two paradigms to efficiently use JAX for DRL systems \cite{PodracerArchitecturesScalable_Hessel.Kroiss.ea_2021}.
In the \textbf{Anakin} architecture, the environment is implemented with JAX and the entire agent-environment loop is compiled using JAX and computed with accelerators.
\textbf{Gymnax}, developed by \citeauthor{GymnaxJAXbasedReinforcement_RobertTjarkoLange_2022}, provides environment implementations in native JAX, and is compatible with the Anakin architecture \cite{GymnaxJAXbasedReinforcement_RobertTjarkoLange_2022}.
However, pure JAX implemented environments are not always feasible, especially when environments involve external services, such as Stella or Unity in their backend.
Alternatively, in the \textbf{Sebulba} architecture, environments run on CPUs, but policies could be compiled and computed on accelerators.
Generated experiences in both architectures can be used to compute gradients directly on accelerators.
Figure \ref{fig:sebulba} illustrates the Sebulba architecture.
\includeimage{sebulba}{\textbf{Sebulba architecture, from \cite{PodracerArchitecturesScalable_Hessel.Kroiss.ea_2021}.}}{
    The environments runs on CPUs.
    Inferences and gradient computations are compiled, optimized and executed on TPUs.
}
\chapter{Problem Definition}

\section{Markov Decision Process and Agent-Environment Interface} \label{sec:markov}
A RL problem is usually represented as a \textbf{Markov Decision Process (MDP)}.
MDP is defined as a four-tuple $(\mathcal{S}, \mathcal{A}, R, P)$.
$\mathcal{S}$ is a set of states that forms the \textbf{state space}.
$\mathcal{A}$ is a set of actions that forms the \textbf{action space};
$P(s' | s, a) = \operatorname{Pr}[ s_{t+1} = s' \mid  s_t = s, a_t = a]$ is the \textbf{transition probability function}.
$R(s, a, s')$ is the \textbf{reward function}.
We use the \textbf{agent-environment interface} (as in Figure \ref{fig:agent_environment_interface}) to solve a problem formulated as an MDP.
The MDP is represented as the \textbf{environment}.
The decision maker that interacts with the environment is called the \textbf{agent}.
At each time step $t$, the agent starts at state $s_t \in \mathcal{S}$, takes an action $a_t \in \mathcal{A}$,
transitions to state $s_{t+1} \in \mathcal{S}$ based on the transition probability function $P(s_{t+1} \mid s_t, a_t)$
and receives a reward $R(s_t, a_t, s_{t+1})$.
These interactions yield a sequence of actions, states, and rewards $s_{0}, a_{0}, r_{1}, s_{1}, a_{1}, r_{2}, \dots$.
We call this sequence a \textbf{trajectory}.
When a trajectory ends at a terminal state $s_T$ at time $t = T$, this sequence is completed and we called it an \textbf{episode}.
Figure \ref{fig:agent_environment_interface} illustrates the interaction between the agent and the environment.

\includeimage[0.7]{agent_environment_interface}{\textbf{The Agent-Environment Interface, from \cite{ReinforcementLearningIntroduction_Sutton.Barto_2018}}.}{}


\section{Policies and Value Functions} \label{sec:policies_and_functions}
At each state $s$, then agent takes an action based on a \textbf{policy} $\pi(a \mid s)$.
This policy represents the conditional probability of the agent taking an action given a state, and $\pi(a \mid s) = Pr[ a_{t} = a \mid  s_t = s]$.
The objective of the agent is to maximize the expected discounted sum of rewards from the current state $s_t$ following the policy $\pi$
\begin{align}
    \text{maximize} ~~~~  & \mathbb{E}_{\pi}\left[ G_t \mid s_t = s \right] , ~~~~ \forall s \in \mathcal{S} \label{eq:maximize_return} \\
    G_t                   & = \sum_{k=0}^{T} \gamma^{k} r_{t+k+1} ~~~~ . 
\end{align}
Here $\gamma$ is the discount factor to favor short-term rewards.
$G$ is the discounted sum of rewards, or, equivalently, the discounted \textbf{return}.
We represent the maximization target above as the \textbf{value function} $V$
\begin{align*}
    V_\pi(s) = \mathbb{E}_{\pi}\left[ G_t \mid s_t = s \right] ~~.
\end{align*}
The value function indicates how good a state is following the policy $\pi$.
Similarly, we define the \textbf{state-action value function}
\begin{align*}
    Q_\pi(s, a) = \mathbb{E}_{\pi}\left[ G_t \mid s_t = s, a_t = a \right]  \\
\end{align*}
that indicates how good a state and action pair is.
We define the $N$-step return as a proxy of the true return, bootstrapped from a value function of a future state
\begin{align*}
    G^N_t = \sum_{k=0}^{N - 1} \gamma^{k} r_{t+k+1} + \gamma^{N} V(s_{t+N}) ~~ .
\end{align*}

\section{Partially Observable Markov Decision Process} \label{sec:pomdp}
A generalization of MDP is a Partially Observable Markov Decision Process (POMDP) \cite{OptimalControlMarkov_Astrom_1965}.
In addition to the four-tuples of MDP, POMDP also defines $\Omega$, a set of observations $o$ that forms the \textbf{observation space}; and $O(o \mid s, a) = \operatorname{Pr}[o_t \mid s_t = s, a_t = a]$, the conditional probability of observing $o_t$ given the last taken action $a_t$ and state $s_t$.
In an agent-environment interface with a POMDP represented environment, the true environment state $s_t$ at each timestep is hidden from the agent and the agent only receives a partial observation $o_t$.

\section{Game Playing}
We can represent board games and video games as POMDPs and solve them by developing an agent.
Many board games, such as Go and chess, are fully observable and we treat them as the special case where $o_t = s_t$.
Video games, however, are partially observable since frames rendered on the screen do not contain all information of the program's running memory.
In Go, chess, and Shogi, the only reward is given from the last timestep based on the game result, and the reward is one if $\{-1, 0, +1\}$.
In Atari games, environments produce intermediate rewards based on game progression, and the scale and density of the rewards varies from game to game.
In all cases, the goal of the agent is to maximize the expected return as described in equation \ref{eq:maximize_return}.

% \section{Models of Environments}
% A \textbf{perfect model} is a subroutine the agent has access to that fully represents the environment dynamics.
% For example, an board game agent has a perfect model if it has access to a full implementation of the game.
% The agent uses the perfect model to try out actions and the perfect model tells the agent the exact state changes after applying the actions.
% In 

% \section{General Game Playing}

% \note{also describes policy $\pi$}
% \note{address this is the most common formulation and how different libraries implement the interface}

% \subsection{Shortcomings of the Agent-Environment Interface for General Game Playing}
% % \note{I'm not sure if I should address these separatly.}
% \subsection{Multi-Agent Games}
% \note{address OpenSpiel's design multiple agents}
% \subsection{Partial Observability}
% \note{address POMDP}
% \subsection{Environment Stochasticity}
% \note{address OpenSpiel's design of random node}
% \subsection{Episodic vs Continuous}
% \note{barely seen in the literature, need more literature review}
% \subsection{Self-Observability}
% % \note{agent needs to be able to observe itself}
% \subsection{Environment Output Structure}
% % The agent-environment interface specifies two return types from the environment, namely the \textbf{observation} and the \textbf{reward}.
% % All environment implementations used in the RL field follow 

\subsection{Our Approach}
\note{(5 pages)}

% The most significant difference of our approach is the separation of data and process.
% In the Agent-Environment Interface, both the agent and the environment are assumed to be stateful, which means they could store and process arbitrary data.

% \subsubsection{Generalized Interaction Interface}
% We propose the \textbf{Generalized Interaction Interface (GII)}.
% We define the \textbf{tape} $E$ as the data storage of the interface, and a \textbf{law} $L$ as a pure function that operates on the tape.
% An instance of such interface could consists of exactly one tape and multiple laws, and we define such an instance a \textbf{universe}.
% A universe \textbf{ticks} by applying the laws on the tape.
% \note{elaborate formally}

% We implement a simplified version of this interface in \textbf{MooZi}.

% % \note{elaborate on the interface}

% \subsubsection{Advantages}
% \note{pure functions are efficient}

\section{Method} \label{sec:method}
\note{(20 - 25 pages)}
\subsection{Design Philosophy}

\subsubsection{Use of Pure Functions}
One of the most notable differences of the MooZi implementation compared to other implementations is the use of pure functions.
In MooZi, we separate the storage of data and the handling of data whenever possible, especially for the parts with heavy computations.
We use \textbf{JAX} and \textbf{Haiku} to implement neural network related modules (\ref{sec:jax_and_podracer}, \cite{HaikuSonnetJAX_Hennigan.Cai.ea_2020,JAXComposableTransformations_JamesBradbury.RoyFrostig.ea_2018}).
These libraries separate the \textbf{specification} and the \textbf{parameters} of a neural network.
The \textbf{specification} of a neural network is a pure function that is internally represented by a fixed computation graph.
The \textbf{parameters} of a neural network includes all learned variables that could be used with the specification to perform a forward pass.
For example, say we have a simple neural network with a single dense layer that does the following
\begin{align*}
    \mathbf{y} = \operatorname{tanh}\left( \mathbf{A}\mathbf{x} + \mathbf{b} \right)
\end{align*}
where $\mathbf{x}$ is the input vector of shape $(n, 1)$, $\mathbf{y}$ is the output vector of shape $(m, 1)$, $\mathbf{A}$ is the learned weights of shape $(m, n)$, and $b$ is the learned bias of shape $(m, 1)$.
We demonstrate how to build this simple network using JAX and Haiku in Algorithm \ref{code:pure}.
We visualize the computation graph of it in Figure \ref{fig:pure}.
\includecode{pure}{
    \textbf{A simple dense layer implemented in JAX and Haiku.}
    The \Verb|model| in the code is the specification of the neural network.
    The \Verb|params| in the code is the parameters of the neural network.
    Only \Verb|params| contains concrete data.
}
\includeimage{pure}{
    \textbf{Computation graph of the simple dense layer in Algorithm \ref{code:pure}.}
    This computation graph show no concrete data, but the data types, shapes, and operators of the layer (\Verb|f32| stands for \textit{single-precision float}).
    To complete a forward pass, we need both concrete neural network parameters ($\mathbf{A}, \mathbf{b}$) and concrete input value ($\mathbf{x}$).
}
Using these pure functions separates the \textit{algorithm} of the agent and the \textit{state} of the agent both conceptually and in implementations.
The \textit{algorithm} part of the agent could be abstracted into a computation graph that could be compiled and optimized using a specialized compiler, such as XLA, for hardware acceleration \ref{sec:jax_and_podracer}.
The \textit{state} part of the agent could be efficiently handled by tools specialized in data manipulations and transferring such as Ray (\ref{sec:ray}).
This way, our system efficiently performs inferences on accelerators and transfers data on CPUs.

\subsubsection{Training Efficiency}
In section \ref{sec:drl_systems} we reviewed common DRL systems in which developers gave training efficiency the highest priority in their system designs.
We also designed our system so that it's efficient and scalable.
Here we describe key features our system has to improve its efficiency.
The first one is system parallelization.
The computation throughput of a single process is simply not enough for DRL systems.
In the published results of MuZero by \cite{MasteringAtariGo_Schrittwieser.Antonoglou.ea_2020}, the agent generated over 20 billion environment frames for training.
Let's do a quick back-of-envelope-calculation for what this means for a non parallelized system.
Consider Gymnax's efficient MinAtar implementation where each environment step takes about 1 millisecond \cite{GymnaxJAXbasedReinforcement_RobertTjarkoLange_2022}.
With a single process, it would take more than 200 days just to step the environment.
As a result, we have to build a distributed system to increase total throughput through parallelism.
The second one is the environment transition speed.
In Atari games, especially Atari games in ALE (\ref{sec:ale}), taking one environment step invokes a full-fledged Atari emulator in the backend and is much more time consuming than neural network inferences.
% On the other hand, neural-networks-based policies can be computed using specialized hardwares such as GPUs and TPUs and are much faster.
Board games, especially those are implemented in performance focus languages , are much faster.
We use MinAtar (\ref{sec:min_atar}) for simpler variants of Atari games, and OpenSpiel for efficient implementations of board games to reduce the time cost on environment transitions.
The third one is neural network inferences used in acting.
DRL systems like IMPALA assume that the policy output can be computed by a single forward pass of a neural network.
However, MuZero's policy not only requires dozens inferences per action taken, but also requires a planner that prepares inputs for the inferences and initiates inferences.
Our system, utilizing JAX and MCTX, handles planning with multiple inferences per action efficiently.

\subsubsection{Understanding is Important}
Machine learning algorithms, especially those involve neural networks, have interpretability issues and sometimes could only be used as ``black boxes'' \cite{ExplainableAIReview_Linardatos.Papastefanopoulos.ea_2021}.
We believe that having a system that we can understand is much more useful for future research than having a system that ``just works''.
Therefore, our project studies the behavior of the system through extensive logging and visualization utilities.
% We will show we use these tools to understand the learned model in section \ref{sec:logging}.

\subsection{Architecture Overview}
In MooZi, we use the \textbf{Ray} library designed by \citeauthor{RayDistributedFramework_Moritz.Nishihara.ea_2018}
for orchestrating distributed processes \cite{RayDistributedFramework_Moritz.Nishihara.ea_2018}.
We also adopt the terminology used by Ray.
In a distributed system with \textbf{centralized control}, a single \textbf{driver} process is responsible for operating all other processes.
Other processes are either \textbf{tasks} or \textbf{actors} .
\textbf{Tasks} are stateless functions that take inputs and return outputs.
\textbf{Actors} are stateful objects that can perform multiple tasks.
In the RL literature, \textbf{actor} is also a commonly used term for describing the process that holds a copy of the network weights and interacts with an environment \cite{SEEDRLScalable_Espeholt.Marinier.ea_2020, IMPALAScalableDistributed_Espeholt.Soyer.ea_2018}.
Even though MooZi does not adopt this concept of a RL actor, we will use the terms \textbf{Ray task} and \textbf{Ray actor} to avoid confusion.
In contrast to distributed systems with \textbf{distributed control}, ray tasks and actors are reactive and do not have busy loops.
The driver controls when a ray task or actor is activated, what data is used as inputs, and where the outputs goes.
The driver orchestrates the data and control flow of the entire system.
Ray tasks and actors merely responded to instructions, process input and return output on command.
We illustrate MooZi's architecture in Figure (\ref{fig:moozi_architecture}).

\includeimage[1]{moozi_architecture}{
    \textbf{MooZi Architecture.}
    The \textit{driver} is the entrance point of the program and is responsible for setting up configurations,
    % spawning other processes as ray actors, and managing data flow among the ray actors \note{add ref section}.
    The \textit{parameter server} stores the latest copy of the network weights and performs batched updates to them (\ref{sec:param_server}).
    The \textit{replay buffer} stores generated trajectories and processes these trajectories into training targets (\ref{sec:replay}).
    A \textit{training worker} is a ray actor responsible for generating experiences by interacting with the environment (\ref{sec:train_rw}).
    A \textit{testing rollout worker} is a ray actor responsible for evaluating the system by interacting with the environment (\ref{sec:test_rw}).
    A \textit{reanalyze rollout worker} is a ray actor that updates search statistics for history trajectories (\ref{sec:re_w}).
}

\subsection{Components}
\subsubsection{Environment Bridges} \label{sec:env_bridge}
Environment bridges unify environments which are defined in different libraries into a shared interface.
In software engineering terms, environment bridges follow the \textbf{bridge design pattern} \cite{BridgePattern__2022}.
In our project we implement environment bridges for three types of environments that are commonly used in RL research: OpenAI Gym, OpenSpiel, and MinAtar \cite{OpenAIGym_Brockman.Cheung.ea_2016,OpenSpielFrameworkReinforcement_Lanctot.Lockhart.ea_2020,MinAtarAtariInspiredTestbed_Young.Tian_2019}.
The bridges wrap these environments into the \textbf{The DeepMind RL Environment API} \cite{DmEnvDeepMind__2022}.
In this format, each environment step outputs a four-tuple $(\text{\Verb|step_type|}, r, \gamma, o)$.
$r, \gamma, o$ are reward, discount, and partial observation respectively.
The \Verb|step_type| is an enumerated value which indicates the type of timestep.
Three possible values of \Verb|step_type| are (1) \Verb|first|, indicating the start of an episode,
(2) \Verb|mid|, indicating an intermediate step, and (3) \Verb|last| indicating the last step of an episode.
Our bridges wrap these environments again to produce a flat dictionary used by MooZi.

The final environments share the same signature as follows:
\begin{itemize}
    \item Inputs
          \subitem $b^{\text{last}}_{t}$: A boolean indicating the episode end.
          \subitem $a_t$: An integer encoding of the action taken.
    \item Outputs
          \subitem $o_t$:
          An N-dimensional array representing the observation of the current timestep as an image
          in the shape $(H, W, C_e)$. $H$ is the height, $W$ is the width, and $C_e$ is the number of channels.
          \subitem $b^{\text{first}}_{t}$: A boolean indicating the episode start.
          \subitem $b^{\text{last}}_{t}$: A boolean indicating the episode end.
          \subitem $r_t$: A float indicating the reward of taking the given action.
          \subitem $m^{A^a}_t$: A bit mask indicating legal action indices. Valid
          action indices are $1$ and invalid actions indices are $0$ (see \ref{sec:a_aug}).
\end{itemize}

All environments are generalized to continuous tasks by passing an addition input $b^\text{last}_t$ to the environment stepping argument.
For an episodic task, the environment is reset internally when $b^{\text{last}}_t$ is \Verb|True|.
The policy still executes for the last environment step, but the resulting action is discarded.
For a continuous task, the environment always step with the latest action and the $b^{\text{last}}_t$ input is ignored.
Algorithm \ref{code:env_interface} demonstrates the unified main loop interface.
\includecode{env_interface}{
    \textbf{Environment Adapter Interface.}
    Both \textit{episodic} environments and \textit{continuous} environments are handled with the same main loop.
}
We also implement a mock environment using the same interface \cite{MockObject__2021}.
A mock environment is initialized with a \textbf{trajectory sample} $\mathcal{T}$, and simulates the environment by outputting step samples one at a time.
An agent can interact with this mock environment as if it were a real environment.
However, the actions taken by the agent do not affect state transitions since they are predetermined by the given trajectory from initialization.
This mock environment is used by the reanalyze rollout workers in section \ref{sec:re_w}.


\subsubsection{Vectorized Environment} \label{sec:vec_env}
We also implement a vectorized environment supervisor that stack multiple individual environments to form a single vectorized environment.
The resulting vectorized environment takes inputs and produces outputs similar to an individual environment but with an additional batch dimension.
For example, an individual environment produces a single frame of shape $(H, W, C)$, while the vectorized environment produces a batched frame of shape $(B, H, W, C)$.
Previously scalar outputs such as reward are also stacked into vectors of size $B$.
Since environment bridges generalize episodic tasks as continuous tasks, we do not need special handling for the first and the last timesteps in the vectorized environment and its main loop looks exactly like that in Algorithm \ref{code:env_interface}.
Using vectorized environments increases the communication bandwidth between the environment and agent and facilitates designing an vectorized agent that processes batched inputs and returns batched actions in one call.

The mock environment described in section \ref{sec:env_bridge} is less trivial to vectorize.
Each mock environment has to be initialized with a trajectory sample.
To vectorize $B$ mock environments, at least $B$ trajectories have to be tracked at the same time.
These $B$ trajectories usually have different length and therefore terminate at different timesteps.
Once one of the mocked trajectories reaches its termination, another trajectory has to fill the slot.
We create a trajectory buffer to address this problem.
When a new trajectory is needed by one of the mocked environments, the buffer replenish it,
so the vectorized mocked environment can process batched interactions like a regular vectorized environment until the trajectory buffer runs out of trajectories.
An external process has to refill the buffer once in a while.
The driver pulls the latest trajectories from the replay buffer and supplies the mock environment's trajectory buffer \note{reference training}.

\subsubsection{Action Space Augmentation} \label{sec:a_aug}
We augment the action space by adding a dummy action $a^\text{dummy}$ indexed at 0.
This dummy action is used to construct history observations when the horizon extends beyond the current timestep.
For example, if the history horizon is 3, we need the last three frames and actions to construct the input observation to the policy.
However, if the current timestep is 0, the agent hasn't taken any actions yet.
We use zeroed frames with the same shape as history frames, and the augmented dummy action as history actions.
Moreover, MooZi's planner (\ref{sec:planner}) does not have access to a perfect model, and it does not know when a node represents a terminal state.
Node expansions do not stop at terminal states and the tree search could simulate multiple steps beyond the end.
Search performed in these invalid subtrees not only wastes precious search budget, but also back-propagates value and reward estimates that are not learned from generated experience.
We address this issue by letting the model learn a policy that always takes the dummy action beyond a terminal state.
This learned dummy action acts as a switch that, once taken, treats all nodes in its subtree as absorbing states and edges that have zero values and rewards respectively.
This discourages the planner to search in invalid regions and improves search performance for near-end game states.
To formally differentiate these two types of action spaces, we denote the original environment action space $\mathcal{A}^e$ and the augmented action space $\mathcal{A}^a$, and
\begin{align*}
    \mathcal{A}^a  & = \mathcal{A}^e \cup a^\text{dummy}  \\
    a_{i}          & = a^\text{dummy} ~~~~ \forall i < 0     & \text{(before the first timestep)}  \\
    a_{i}          & = a^\text{dummy} ~~~~ \forall i \geq T  & \text{(after the last timestep)}  \\
\end{align*}
Notice that the environment terminates at timestep $T$ so the last effective action taken by the agent is $a_{T-1}$.

\subsubsection{History Stacking} \label{sec:history_stacking}
In fully observable environments, the state $s_t$ at timestep $t$ observed by the agent entails sufficient information about the future state distribution.
However, for partially observable environments, this does not hold.
The optimal policy might not be representable by a policy $\pi(a \mid o_t)$ that only takes into account the most recent partial observation $o_t$.
Most Atari games are such partially observable environments.
In DQN, \citeauthor{PlayingAtariDeep_Mnih.Kavukcuoglu.ea_2013} alleviated this problem by augmenting the inputs of the policy network from a single frame observation to a stacked history of four frames so that the policy network had a signature of $\pi(a \mid o_{t-3}, o_{t-2}, o_{t-1}, o_t)$ (\ref{sec:dqn}, \cite{PlayingAtariDeep_Mnih.Kavukcuoglu.ea_2013}).
AlphaZero and MuZero use not only a stacked history of environment frames, but also a history of past actions.
MooZi uses the last $L$ environment frames and taken actions, so the signature of the learned model through the policy head of the prediction function is $\mathbf{p} = f(a \mid o_{t - L + 1}, \dots, o_t, a_{t - L}, \dots, a_{t-1})$.
The greater $L$ is, the better the stacked observation represents a full state.
In a deterministic environment with a fixed starting state, the stacked history represents a full environment state when $L = \infty$.
On the other hand, $L = 1$ is sufficient for fully-observable perfect information environments.

The exact process of creating the model input by stacking history frames and actions is as follows:
\begin{enumerate}
    \item Prepare $L$ saved environment frames of shape $(L, H, W, C_e)$.
    \item Stack the $L$ dimension with the environment channels dimension $C_e$, resulting in shape $(H, W, L * C_e)$
    \item Prepare saved $L$ past actions of shape $(L)$, encoded as integers.
    \item One-hot encode the actions as shape $(L, |\mathcal{A}^a|)$.
    \item Normalize the action planes by the number of actions $|\mathcal{A}^a|$, shape remains the same.
    \item Stack the $L$ axis with the action axis, now shape $(L * |\mathcal{A}^a|)$.
    \item Tile action planes $(L * |\mathcal{A}^a|)$ along the $H$ and $W$ dimensions, now shape $(H, W, L * |\mathcal{A}^a|)$
    \item Stack the environment planes and actions planes, now shape $(H, W, L * (C_e + |\mathcal{A}^a|))$
    \item The history is now represented as an image with height of $H$, width of $W$, and $L * (C_e + |\mathcal{A}^a|)$ channels
\end{enumerate}

To process batched inputs from vectorized environments described in \ref{sec:vec_env}, all operations above are performed with an additional batch dimension $B$, yielding the final output with the shape $(B, H, W, L * (C_e + |\mathcal{A}^a|))$.
We denote the channels of the final stacked history as $C_h = L * (C_e + |\mathcal{A}^a|)$, where the subscript $h$ means the channel dimension for the representation function $h$.

Figure \ref{fig:stacking} illustrates this process with an example.
\includeimage[1]{stacking}{
    An example of history stacking.
    \textit{History}: Partial observations and actions from the last 3 timesteps ($L = 3$). Actions are integers and observations are images with 2 channels each.
    \textit{One-hot Actions}: One-hot encodes $L$ history actions into vectors.
    \textit{Normalize Actions}: Diviv
    \textit{Actions to Planes}: One-hot encodes actions into feature planes that has the same resolution (i.e., same width and height) as the observations, $|\mathcal{A}^a| = 2$.
    \textit{Stack Planes}: Stack all planes together, creating an image with 12 channels and the same resolution as the observations.
}



\subsubsection{Planner} \label{sec:planner}
The planner is the component that decides what actions to take based on the latest observations.
% The planner prepares inputs for the search, performs the search, collects search statistics, sends an action to the environment.
We use the MuZero variant of MCTS described in \ref{sec:mcts} and \ref{sec:muzero} with the help from \textbf{MCTX} by \citeauthor{POLICYIMPROVEMENTPLANNING_Danihelka.Guez.ea_2022} \cite{POLICYIMPROVEMENTPLANNING_Danihelka.Guez.ea_2022}.
The planner $\mathcal{P}$ takes a stacked history as its input (\ref{sec:history_stacking}), performs a search, collects search statistics, and outputs a action and search statistics
\begin{align*}
    a_t, v^*_t, \mathbf{p}^*_t = \mathcal{P}(o_{t - L + 1}, \dots, o_t, a_{t - L}, \dots, a_{t - 1}) ~~ .
\end{align*}
Here $v^*_t$ is the search-updated value estimate of the root, $\mathbf{p}^*_t$ is the search-updated action visits at the root, and $a_t$ is action to take.

\subsubsection{MooZi Neural Network} \label{sec:nn}
\note{neural network specifical should go into experiment section}
We used JAX, and Haiku to build the neural network \cite{HaikuSonnetJAX_Hennigan.Cai.ea_2020,CompilingMachineLearning_Frostig.Johnson.ea_2019,JAXComposableTransformations_JamesBradbury.RoyFrostig.ea_2018}.
We consulted other open-source projects that use neural networks to play games \cite{MuZeroGeneral_Duvaud.AureleHainaut_2022, MasteringAtariGames_Ye.Liu.ea_2021, AcceleratingSelfPlayLearning_Wu_2020}.
We implemented the neural-network model with two different architectures in our project, one is multilayer-perceptron-based and the other one is residual-blocks-based \cite{DeepResidualLearning_He.Zhang.ea_2016}.
We primarily used residual-blocks-based model for experiments so we will describe the architecture in full details here.

Similar to MuZero described in section \ref{sec:muzero}, the model had the representation function, the dynamics function, and the dynamics function.
Additionally, we also trained the MooZi model with a self-consistency loss similar to that described by \citeauthor{MasteringAtariGames_Ye.Liu.ea_2021} and \citeauthor{VisualizingMuZeroModels_deVries.Voskuil.ea_2021} \cite{MasteringAtariGames_Ye.Liu.ea_2021,VisualizingMuZeroModels_deVries.Voskuil.ea_2021}.
We used an additional function, named as the \textbf{projection function} for this purpose.
The learned model was used for two purposes during tree searches.
The first one was to construct the root nodes using the representation function and the prediction function.
We call this process the \textbf{initial inference}.
The second one was to create edges and child nodes for a given node and action using the dynamics function and the prediction function.
We call this process the \textbf{recurrent inference}.
\note{This terminology is aligned with MuZero's pseudo-code and MCTX's real code}

We implemented residual blocks using the same specification as \citeauthor{DeepResidualLearning_He.Zhang.ea_2016} \cite{DeepResidualLearning_He.Zhang.ea_2016}.
One residual block was defined as follows:
\begin{itemize}
    \item input $x$
    \item save a copy of $x$ to $x'$
    \item apply a 2-D padded convolution on $x$, with kernel size 3 by 3, same channels
    \item apply batch normalization on $x$
    \item apply relu activation on $x$
    \item apply a 2-D padded convolution on $x$, with kernel size 3 by 3, same channels
    \item apply batch normalization on $x$
    \item add $x'$ to $x$
    \item apply relu activation on $x$
\end{itemize}

The representation function $h$ is parametrized as follows:
\begin{itemize}
    \item input $x$ of shape $(H, W, C_h)$
    \item apply a 2-D padded convolution on $x$, with kernel size 3 by 3, 32 channels
    \item apply batch normalization on $x$
    \item apply relu activation on $x$
    \item apply 6 residual blocks with 32 channels on $x$
    \item apply a 2-D padded convolution on $x$, with kernel size 3 by 3, 32 channels
    \item apply batch normalization on $x$
    \item apply relu activation on $x$
    \item output the hidden state head $x_s$ % of shape $(H, W, 32)$
\end{itemize}

The prediction function $f$ is parametrized as follows:
\begin{itemize}
    \item input $x$ % of shape $(H, W, 32)$
    \item apply a 2-D padded convolution on $x$, with kernel size 3 by 3, 32 channels
    \item apply batch normalization on $x$
    \item apply relu activation on $x$
    \item apply 1 residual block with 32 channels on $x$
    \item flatten $x$
    \item apply 1 dense layer with output size of 128 to obtain the value head $x_v$
    \item apply batch normalization on $x_v$
    \item apply relu activation on $x_v$
    \item apply 1 dense layer with output size of $Z$ on $x_v$
    \item apply 1 dense layer with output size of 128 to obtain the policy head $x_p$
    \item apply batch normalization on $x_p$
    \item apply relu activation on $x_p$
    \item apply 1 dense layer with output size of $A^a$ on $x_p$
    \item output the value head $x_v$ and the policy head $x_p$
\end{itemize}

The dynamics function $g$ is parametrized as follows:
\begin{itemize}
    \item input $x$ of shape $(H, W, 32)$, $a$ as an integer
    \item encode $a$ as action planes of shape $(H, W, A)$ (described in \ref{sec:history_stacking})
    \item append $a$ to $x$

    \item apply a 2-D padded convolution on $x$, with kernel size 3 by 3, 32 channels
    \item apply batch normalization on $x$
    \item apply relu activation on $x$

    \item apply 1 residual block with 32 channels on $x$

    \item apply 1 residual block with 32 channels on $x$ to obtain the hidden state head $x_s$
    \item apply a 2-D padded convolution on $x_s$, with kernel size 3 by 3, 32 channels
    \item apply batch normalization on $x_s$
    \item apply relu activation on $x_s$

    \item apply 1 dense layer with output size of 128 on $x$ to obtain the reward head $x_r$
    \item apply batch normalization on $x_r$
    \item apply relu activation on $x_r$
    \item apply 1 dense layer with output size of $Z$ on $x_r$

    \item output the hidden state head $x_s$ and the reward head $x_r$
\end{itemize}

For convenience, the output specification is the same for both the initial inference and the recurrent inference.
They both produce a tuple of $(\mathbf{x}, v, \hat{r}, \mathbf{p})$, where $\mathbf{x}$ is the hidden state, $v$ is the value prediction, $\hat{r}$ is the reward prediction, and $\mathbf{p}$ is the policy prediction.
For the initial inference,
\begin{itemize}
    \item input features $\psi_t = (o_{t - L + 1}, \dots, o_t, a_{t - L}, \dots, a_{t -1})$
    \item obtain $\mathbf{x}_t^0 = h(\psi_t)$
    \item obtain $v^0_t, \mathbf{p}^0_t = f(\mathbf{x}_t^0)$
    \item set $\hat{r}_t^0 = 0$
    \item return $(\mathbf{x}^0_t, v_t^0, \hat{r}_t^0, \mathbf{p}^0_t)$
\end{itemize}
For the recurrent inference,
\begin{itemize}
    \item input features $\mathbf{x}_t^i, a_t^i$
    \item obtain $\mathbf{x}_t^{i+1}, \hat{r}_t^{i+1} = g(\mathbf{x}_t^i, a_t^i)$
    \item obtain $v^{i+1}_t, \mathbf{p}^{i+1}_t = f(\mathbf{x}_t^{i+1})$
    \item return $(\mathbf{x}^{i+1}_t, v_t^{i+1}, \hat{r}_t^{i+1}, \mathbf{p}^{i+1}_t)$
\end{itemize}

Moreover, we applied the invertible transformation \( \phi \) described in section \ref{sec:scalar_transform} to both the scalar reward targets and scalar value targets to create categorical representations with the same support size.
The support we used for the transformation were integers from the interval \( [-5, 5] \), with a total size of 11.
Scalars were first transformed using \( \phi \), then converted to a linear combination of the nearest two integers in the support.
For example, for scalar \(\phi(x) = 1.3\), the nearest two integers in the support are $1$ and $2$, and the linear combination is \( \phi(x) = 1 * 0.7 + 2 * 0.3 \), which means the target of this scalar is $0.7$ for the category $1$, and $0.3$ for the category $2$.
We denote $\Phi$ for this process of applying $\phi$ then categorizing the resulting value into a support $Z$.
Using the same example that $\phi(x) = 1.3$, assume the support is $Z = [-2, -1, 0, 1, 2], |Z| = 5$,
then $\Phi(x) = [0, 0, 0, 0.7, 0.3]$, and $\Phi(x) \cdot Z = \phi(x) = 1.3$.
For training, the value head and the reward head first produced estimations as logits of size $|Z|$.
These logits were aligned with the scalar targets to produce categorization loss as described in the \ref{sec:loss}.
For acting, the neural network additionally applied the softmax function to the logits to generated a distribution over the support.
The linear combination of the distribution and their corresponding integer values were computed and fed through the inverse of the transformation, namely \( \phi^{-1}\), to produce scalar values.
This means from the perspective of the planner (\ref{sec:planner}), the scalar estimations made by the model were in same shape and scale as those produced by the environment.

\subsubsection{Training Targets Generation} \label{sec:targets}
% The agent interacted with the environment by taking actions.
At each timestep $t$, the environment provides a tuple of data as described in section (\ref{sec:env_bridge}).
The agent interacts with the environment by performing a tree search and taking action $a_t$.
The search statistics of the tree search were also saved, including the updated value estimate of the root action $\hat{v}_t$,
and the updated action probability distribution $\hat{p}_t$.
These completes one \textbf{step sample} $\mathcal{T}_t$ for timestep $t$, which is a tuple of $(o_t, a_t, b^{\text{first}}_{t}, b^{\text{last}}_{t}, r_t, m^{A_a}_t, \hat{v}_t, \hat{p}_t)$.
Once an episode concludes ($b^{\text{last}}_{T} = 1)$, all recorded step samples are gathered and stacked together.
This yields a final trajectory sample $\mathcal{T}$ that has a similar shape to a step sample but with an extra batch dimension with the size of $T$.
For example, $o_t$ is stacked from shape $(H, W, C_e)$ to shape $(T, H, W, C_e)$.
The training workers described in \ref{sec:train_rw} generate trajectories this way.
The reanalyze rollout workers generate trajectories with the same signature, but through statistics update described in using a vectorized mocked environment (see \ref{sec:reanalyze} and \ref{sec:vec_env}).

Each trajectory sample with $T$ step samples were processed into $T$ training targets.
For each training target at timestep $i$, we create a training target as follows:
\begin{itemize}
    \item Observations $o_{i - L + 1}, \dots, o_{i + 1}$ where $H$ is the history stacking size.
          The first $H$ observations were used to create policy inputs as described in \ref{sec:history_stacking},
          and the pair of observation $o_{i}, o_{i+i}$ were used to compute self-consistency loss described in \ref{sec:loss}.

    \item Actions $a_{i - L}, \dots, a_{i + K - 1}$.
          Similarly, The first $H$ actions were used for policy input and the pair of actions at $(a_{i - 1}, a_{i})$ were used for self-consistency loss.
          The actions $a_{i}, \dots, a_{i + K - 1}$ were used to unroll the model during the training for $K$ steps.

    \item Rewards $r_{i + 1}, \dots, r_{i + K}$ as targets of the reward head of the dynamics function.

    \item Action probabilities $\mathbf{p}^*_{i}, \dots, \mathbf{p}^*_{i + K}$ from the statistics of $K + 1$ search trees.

    \item Root values $v^*_i, \dots, v^*_{i + K}$, similarly, from the statistics of $K + 1$ search trees.

    \item N-step return $G^N_{i}, \dots, G^N_{i + K}$.
          Each N-step return was computed based on the formula
          \begin{align*}
              G^N_{t} = \sum_{i = 0}^{N - 1}{\gamma^i r_{t+i+1}} + \gamma^Nv^*_{t + N}
          \end{align*}

    \item Importance sampling ratio $\rho = 1$. Placeholder value for future override based on replay buffer sampling weights (see \ref{sec:replay}).
\end{itemize}
Training targets were computed with minimum information necessary to be used in the loss function (\ref{sec:loss}) so that the precomputed training targets take up the least memory.

\subsubsection{Loss Computation} \label{sec:loss}
Our loss function is similar to that of \ref{sec:muzero}, but with additional self-consistency loss, terminal action loss, and value loss coefficient
\begin{align*}
    \mathcal{L}_{t}(\theta)
      & =
    \Bigg[
    \underbrace{\mathcal{L}^p(\mathbf{p}^*_t, \mathbf{p}^0_t) + \frac{1}{K}\sum_{k=1}^{K} \mathcal{L}^{p}\left(\mathbf{p}^*_{t+k}, \mathbf{p}_{t}^{k}\right)}_{\circled{1}}  \\
      & +
    \underbrace{c^v\left(\mathcal{L}^v(G^N_{t}, v^*_t) + \frac{1}{K}\sum_{k=1}^{K} \mathcal{L}^{v}\left(G^N_{t+k}, v^*_{t+k}\right) \right)}_{\circled{2}}  \\
      & +
    \underbrace{\sum_{k=1}^{K} \mathcal{L}^{r}\left(\hat{r}_t^k, r_{t+k}\right)}_{\circled{3}}
    +
    \underbrace{c^{s}\mathcal{L}^s_t(\mathbf{x}^1_t, \mathbf{x}^0_{t+1})}_{\circled{4}}  \\
      & +
    \underbrace{c^{L_2}\|\theta\|^{2}}_{\circled{5}}
    \Bigg] \cdot \rho
    \\
\end{align*}
To compute terms used in the loss function, we use the history observations \(o_{t-L+1}, \dots, o_t\) and history actions \(a_{t-L}, \dots, a_{t -1}\) to reconstruct the stacked frames as the input of the initial inference (\ref{sec:history_stacking}).
We apply the initial inference to obtain $\mathbf{p}^0_t, v^0_t, \mathbf{x}^0_t$.
We apply $K$ consecutive recurrent inferences using actions \(a_t, \dots, a_{t+K - 1}\) to obtain \(\mathbf{p}^1_t, \dots, \mathbf{p}^K_t, v^1_t, \dots, v^K_t, \mathbf{x}^1_t, \dots, \mathbf{x}^K_t\).
The policy loss \circled{1} is the standard categorization loss using cross-entropy
\begin{align*}
    \mathcal{L}^p(\mathbf{p}, \mathbf{q}) = - \sum_{p \in \mathbf{p}, q \in \mathbf{q}} p \log{q}
\end{align*}
The policy targets $\mathbf{p}^*_{t+i} (i = 0, 1, \dots, K)$ are action visits at the root of $K+1$ searches performed in the game (\ref{sec:targets}).
To compute the value loss $\circled{2}$ and the reward loss $\circled{3}$, we apply the scalar transformation $\Phi$ (\ref{sec:scalar_transform}) that converts scalar values to categorizations,
and use the same cross-entropy categorization loss
\begin{align*}
    \mathcal{L}^v(p, q)  & = \mathcal{L}^r(p, q) = - \sum_{p \in \Phi(p), q \in \Phi(q)} p \log{q}  \\
\end{align*}
To compute the self-consistency loss $\circled{4}$, we reconstruct the initial inference for the next timestep \(o_{t-L+2}, \dots, o_{t+1}, a_{t-L+1}, \dots, a_{t}\), and compute the cosine distance between the projected one-step hidden state $\varrho(\mathbf{x}^1_t)$ of timestep $t$ and the initial hidden state $\mathbf{x}^0_{t+1}$ of the next timestep $t+1$.
Formally,
\begin{align*}
    \text{cosine distance} ~ (\mathbf{a}, \mathbf{b})
                                                                     & = 1 - \frac{\mathbf{a} \cdot \mathbf{b}}{\|\mathbf{a}\|\|\mathbf{b}\|}  \\
    \circled{4} = \mathcal{L}^s(\mathbf{x^1_t}, \mathbf{x}^0_{t+1})  & = 1 - \frac{\varrho(\mathbf{x}^1_t) \cdot \mathbf{x}^0_{t+1}}{\|\varrho(\mathbf{x}^1_t)\| \| \mathbf{x}^0_{t+1}|\|}
\end{align*}
Figure \ref{fig:consistency_loss} illustrates the intuition behind this loss.
\includeimage{consistency_loss}{
    \textbf{Self-consistency Loss Computation}.
    The hidden state $\mathbf{x}^1_t$ after projection should be similar to the hidden state $\mathbf{x}^0_{t+1}$.
    We assume the next timestep has more information, so we stop gradient from $\mathbf{x^0}_{t+1}$ to push the
    representation of the previous timestep towards the next timestep.
}
\circled{5} is a standard $L_2$ regularization loss to prevent network from overfitting,
and coefficient $c^{L_2}$ is used to control the strength of this regularization.
The overall loss of a training target is scaled by its importance sampling ratio.
We also use the gradient scaling described by \citeauthor{MasteringAtariGo_Schrittwieser.Antonoglou.ea_2020}
that halves the gradient at the beginning of each dynamics function call \cite{MasteringAtariGo_Schrittwieser.Antonoglou.ea_2020}.

\subsubsection{Updating the Parameters}
We use a standard \textbf{Adam} optimizer developed by \citeauthor{AdamMethodStochastic_Kingma.Ba_2017} \cite{AdamMethodStochastic_Kingma.Ba_2017}.
We also clip the gradient as described by \citeauthor{DifficultyTrainingRecurrent_Pascanu.Mikolov.ea_} \cite{DifficultyTrainingRecurrent_Pascanu.Mikolov.ea_}.
The dynamics function $g$ in our learned model is essentially an RNN, so we expect this gradient clipping trick to have a similar effect in our model.
\textbf{Optax}, developed by \citeauthor{OptaxComposableGradient_MatteoHessel.DavidBudden.ea_2020}, is a library for gradient manipulations implemented in JAX \cite{OptaxComposableGradient_MatteoHessel.DavidBudden.ea_2020}.
We use Optax's implementation for both the Adam optimizer and the gradient clipper.
Moreover, we also use a target network that was used in DQN to stabilize training \cite{PlayingAtariDeep_Mnih.Kavukcuoglu.ea_2013}.

\subsubsection{Reanalyze} \label{sec:reanalyze}
In \ref{sec:muzero_reanalyze}, we reviewed \textbf{MuZero Reanalyze}.
In our project, we also implement a type of worker process that re-runs search on old trajectories with the latest neural network parameters.
Given a trajectory sample $\mathcal{T}$, for each timestep $t$ in the trajectory, the reanalyze process is as follows
\begin{itemize}
    \item Use observations $(o_{t - T + 1}, \dots, o_{t})$ and actions $(a_{t - T}, \dots, a_{t - 1})$ to reconstruct the planner input.
    \item Feed the planner $\mathcal{P}$ with the reconstructed input, obtaining the update action $\tilde{a_t}$, the updated policy target at the root $\tilde{\mathbf{p}^*_t}$, and the updated value target at the root $\tilde{v^*_t}$.
    \item Discard the updated action $\tilde{a_t}$ since the action that got executed in the environment has to be the old action $a_t$ to keep the trajectory consistent.
    \item Replace the old policy target $\mathbf{p}^*_t$ with the updated policy target $\tilde{\mathbf{p}^*_t}$.
    \item Replace the old value target $v^*_t$ with the updated policy target $\tilde{v^*_t}$.
\end{itemize}
Once the entire trajectory $\mathcal{T}$ is processed, we obtain an updated trajectory $\tilde{\mathcal{T}}$ in which only the value targets and policy targets are replaced.

% \subsubsection{Rollout Workers} \label{sec:rw}
% \textbf{Rollout workers} are ray actors that store copies of environments or history trajectories and generated data by evaluating policies and interacting with the environments or history trajectories (also see \ref{sec:ray}).
% A rollout worker does not inherently serve a specific purpose in the system and its behavior is mostly determined by the configuration of it.
% There are three main types of rollout workers used in MooZi: \textbf{Training Worker}, \textbf{interaction testing rollout worker}, and \textbf{reanalyze rollout worker}.

\subsubsection{Training Worker} \label{sec:train_rw}
The main goal of \textbf{training workers} is to generate trajectories by interacting with environments for training purposes.
For each worker, a vectorized environment is created as described in \ref{sec:vec_env}, a history stacker is created as described in \ref{sec:history_stacking}, and a planner was created using MCTS configurations as described in \ref{sec:planner}.
Each worker also has a delayed copy of the parameters similar to that in IMPALA (\ref{sec:impala} and \cite{IMPALAScalableDistributed_Espeholt.Soyer.ea_2018}).
Step samples and trajectory samples are collected as the planners giving actions and the vectorized environments taking the actions.
Each worker is allocated with one CPU and a fraction of a GPU (usually $10\% \tilde 20\%$ of a GPU) so neural network inferences could be done on GPU.
Collected trajectory samples are returned as the final output of one run of the worker.
See \ref{code:interaction_training_rollout_worker} for the pseudo-code of this process.
The planner of these workers are configured to have more exploration to generate more diverse data.
The exploration is encouraged by setting a greater \Verb|dirichlet_fraction|, a greater \Verb|dirichlet_alpha|, and a greater \Verb|temperature|. \note{move these parameters into planner section}
% \includecode{interaction_training_rollout_worker}{Training Worker}

\subsubsection{Testing Worker} \label{sec:test_rw}
The main goal of \textbf{testing workers} is to generate trajectories by interacting with environments for evaluation.
These workers are similar to training workers and they hold the same type of data.
The differences are: testing rollout workers only use a single environment, have less GPU allocation, and only ran once every other $n$ training steps, where $n$ is a configurable number (usually 5).

\subsubsection{Reanalyze Worker} \label{sec:re_w}
The main goal of \textbf{reanalyze workers} is to update search statistics using the reanalyze process described in \ref{sec:reanalyze}, and push updated trajectories to the replay buffer.

\subsubsection{Replay Buffer} \label{sec:replay}
The \textbf{replay buffer} processes trajectories into training targets and samples trajectories or training targets.
Since most training targets are expected to be sampled more than once,
the replay buffer precomputes the training targets for all received trajectory samples in the replay buffer with the process described in \ref{sec:targets}.
The replay buffer also computes the value difference $\delta$ for each target,
which is the difference between the predicted value from the search, and the bootstrapped N-step return (\ref{sec:targets})
\begin{align*}
    \delta_i = | v^*_i - G^N_i |
\end{align*}
We implemented three modes of sampling: \textbf{uniform}, \textbf{proportional}, and \textbf{rank-based}.
In uniform sampling, every training target has equal probability of being drawn.
The proportional sampling and rank-based sampling follows the same formula described by \citeauthor{PrioritizedExperienceReplay_Schaul.Quan.ea_2016} \cite{PrioritizedExperienceReplay_Schaul.Quan.ea_2016}.
However, instead of one-step temporal difference error, we use the $\delta$ error we described above.
For each training target $i$, the replay buffer also computes the importance sampling ratio $\rho(i)$ based on the probability $P(i)$ of it being drawn
\begin{align*}
    \rho_{i}= \frac{1}{N \cdot P(i)}
\end{align*}
Since the probabilities of targets depends on other targets as well, the importance sampling ratio of targets are not static,
and have to be recomputed each time a batch is sampled from the replay buffer.

\subsubsection{Parameter Server} \label{sec:param_server}
The parameter server holds the central copy of the neural network parameters and updates the parameters.
Once a batch of training targets is received by the parameter server, the loss is computed as described in \ref{sec:loss}.

% \begin{itemize}
%     \item stores a copy of the neural network specification
%     \item stores the latest copy of neural network parameters
%     \item stores the loss function
%     \item stores the training state
%     \item computes forward and backward passes and updates the parameters
% \end{itemize}

\section{System in Action}
% Now we look at how the components run and interact with each other during training.
\subsection{The Driver}
\includecode{driver}{\textbf{The driver.}}
Algorithm \ref{code:driver} is the driver 
The driver starts by initializing all rollout workers, a parameter server, and a replay buffer.
At the beginning of a training step, the driver performs lightweight tasks of all processes such as synchronizing parameters.
During the training step, all processes perform their heavyweight tasks.
Rollout workers interact with environments, the parameter server computes gradients, and the replay buffer process trajectories into training targets.
The method calls made by the driver do not block.
They schedule call events and return immediately rather than waiting for the methods to finish.
The immediate return values of the calls are \textit{promises} managed by Ray \cite{FuturesPromises__2022}.
Actors execute their scheduled method calls sequentially once their concrete inputs are ready.
\includeimage[1]{training_step}{
    \textbf{Timeline of training steps.}
    The red bar indicates a synchronization barrier.
    The duration of each training step is decided by the last finished task.
}

\section{Logging and Visualization} \label{sec:logging}
MooZi incorporates extensive logging and visualization utilities to help users understand its behavior better.
All distributed process contains a dedicated log file that records all events within the process.
Figure \note{add figure here} shows an example of the log files.
MooZi uses \textbf{TensorBoard} to log informative scalars and vectors, including average returns, distribution of importance sampling ratios, replay buffer saturation status, gradient sizes, and much more \cite{TensorFlowLargeScaleMachine_Abadi.Agarwal.ea_}.
Figure \note{add figure here} shows a screenshot of the TensorBoard.
MooZi also provides utilities to visualize the behavior of the algorithm.
Testing workers use the GIF maker tool to create animated records of evaluations.
Figure \note{add figure here} shows an example of the GIF tiled as a sprite image.


\includeimage[1]{experiments_breakout_w_wo_sticky_actions}{
    \textbf{Sticky Actions}
}

\includeimage[1]{experiments_minatar_moozi_vs_ppo}{
    \textbf{Sticky Actions}
}

\includeimage[1]{experiments_space_invaders_vs_simulations}{
    \textbf{Sticky Actions}
}

\section{Things that went wrong}
% batch norm and importance sampling

\section{Conclusion}
\note{(3 pages)}
\subsection{Future Work}
\note{(1 page)}

\printbibliography

\end{document}

% \subsection*{Artificial Intelligence}

% % Artificial Intelligence (AI) is a branch of computer science that emphasizes the use of computer algorithms to solve problems likes humans.
% To define the goals and methods of Artificial Intelligence (AI), we first need to define what is \textit{intelligence}.
% Though there is no consensus on the exact definition of intelligence, here we adopt the definition by John McCarthy:
% \begin{quote}
%     Intelligence is the computational part of the ability to achieve goals in the world.
% \end{quote}
% The goal of AI is to develop computer algorithms that can solve problems and achieve goals in complex environments.
% A diverse range of methods are designed as AI algorithms since the term has been coined.
% \note{A simple list of AI algorithms, such as A*, symbolic}


% \subsection*{Game Artificial Intelligence}
% Perfect vs in-Perfect information
% \subsection*{Planning}
% lookahead search: A*, DFS, BFS
