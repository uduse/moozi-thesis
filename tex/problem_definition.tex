\chapter{Problem Definition}

\section{Markov Decision Process and Agent-Environment Interface} \label{sec:markov}
A RL problem is usually represented as a \textbf{Markov Decision Process (MDP)}.
MDP is defined as a four-tuple $(\mathcal{S}, \mathcal{A}, R, P)$.
$\mathcal{S}$ is a set of states that forms the \textbf{state space}.
$\mathcal{A}$ is a set of actions that forms the \textbf{action space};
$P(s' | s, a) \doteq \operatorname{Pr}[ s_{t+1} = s' \mid  s_t = s, a_t = a]$ is the \textbf{transition probability function}.
$R(s, a, s')$ is the \textbf{reward function}.
We use the \textbf{agent-environment interface} (as in Figure \ref{fig:agent_environment_interface}) to solve a problem formulated as an MDP.
The MDP is represented as the \textbf{environment}.
The decision maker that interacts with the environment is called the \textbf{agent}.
At each time step $t$, the agent starts at state $s_t \in \mathcal{S}$, takes an action $a_t \in \mathcal{A}$,
transitions to state $s_{t+1} \in \mathcal{S}$ based on the transition probability function $P(s_{t+1} \mid s_t, a_t)$
and receives a reward $R(s_t, a_t, s_{t+1})$.
These interactions yield a sequence of actions, states, and rewards $s_{0}, a_{0}, r_{1}, s_{1}, a_{1}, r_{2}, \dots$.
We call this sequence a \textbf{trajectory}.
When a trajectory ends at a terminal state $s_T$ at time $t = T$, this sequence is completed and we called it an \textbf{episode}.
Figure \ref{fig:agent_environment_interface} illustrates the interaction between the agent and the environment.

\includeimage[0.7]{agent_environment_interface}{\textbf{The Agent-Environment Interface, from \cite{ReinforcementLearningIntroduction_Sutton.Barto_2018}}.}{}


\section{Policies and Value Functions} \label{sec:policies_and_functions}
At each state $s$, the agent takes an action based on a \textbf{policy} $\pi(a \mid s)$.
This policy represents the conditional probability of the agent taking an action given a state, $\pi(a \mid s) \doteq Pr[ a_{t} = a \mid  s_t = s]$.
The objective of the agent is to maximize the expected discounted sum of rewards from the current state $s_t$ following the policy $\pi$
\begin{align}
    \text{maximize} ~~~~  & \mathbb{E}_{\pi}\left[ G_t \mid s_t = s \right] , ~~~~ \forall s \in \mathcal{S} \label{eq:maximize_return} \\
    G_t                   & \doteq \sum_{k=0}^{T} \gamma^{k} r_{t+k+1} ~~~~ . 
\end{align}
Here $\gamma$ is the discount factor to favor short-term rewards.
$G$ is the discounted sum of rewards, or, equivalently, the discounted \textbf{return}.
We represent the maximization target above as the \textbf{value function} $V$
\begin{align*}
    V_\pi(s) \doteq \mathbb{E}_{\pi}\left[ G_t \mid s_t = s \right] ~~.
\end{align*}
The value function indicates how good a state is when following the policy $\pi$.
Similarly, we define the \textbf{state-action value function}
\begin{align*}
    Q_\pi(s, a) \doteq \mathbb{E}_{\pi}\left[ G_t \mid s_t = s, a_t = a \right]  \\
\end{align*}
that indicates how good a state and action pair is.
We define the $N$-step return as a proxy of the true return, bootstrapped from a value function of a future state
\begin{align*}
    G^N_t \doteq \sum_{k=0}^{N - 1} \gamma^{k} r_{t+k+1} + \gamma^{N} V(s_{t+N}) ~~ .
\end{align*}

\section{Partially Observable Markov Decision Process} \label{sec:pomdp}
A generalization of MDP is a Partially Observable Markov Decision Process (POMDP) \cite{OptimalControlMarkov_Astrom_1965}.
In addition to the four-tuples of MDP, POMDP also defines $\Omega$, a set of observations $o$ that forms the \textbf{observation space}; and $O(o \mid s, a) \doteq \operatorname{Pr}[o_t \mid s_t = s, a_t = a]$, the conditional probability of observing $o_t$ given the last taken action $a_t$ and state $s_t$.
In an agent-environment interface with a POMDP represented environment, the true environment state $s_t$ at each timestep is hidden from the agent and the agent only receives a partial observation $o_t$.

\section{Game Playing}
We can represent board games and video games as POMDPs and solve them by developing an agent.
Many board games, such as Go and chess, are fully observable and we treat them as the special case where $o_t = s_t$.
Video games, however, are partially observable since frames rendered on the screen do not contain all information of the program's running memory.
In Go, chess, and Shogi, the only reward is given from the last timestep based on the game result, and the reward is one if $\{-1, 0, +1\}$.
In Atari games, environments produce intermediate rewards based on game progression, and the scale and density of the rewards varies from game to game.
In all cases, the goal of the agent is to maximize the expected return as described in Equation \ref{eq:maximize_return}.

% \section{Models of Environments}
% A \textbf{perfect model} is a subroutine the agent has access to that fully represents the environment dynamics.
% For example, an board game agent has a perfect model if it has access to a full implementation of the game.
% The agent uses the perfect model to try out actions and the perfect model tells the agent the exact state changes after applying the actions.
% In 

% \section{General Game Playing}

% \note{also describes policy $\pi$}
% \note{address this is the most common formulation and how different libraries implement the interface}

% \subsection{Shortcomings of the Agent-Environment Interface for General Game Playing}
% % \note{I'm not sure if I should address these separatly.}
% \subsection{Multi-Agent Games}
% \note{address OpenSpiel's design multiple agents}
% \subsection{Partial Observability}
% \note{address POMDP}
% \subsection{Environment Stochasticity}
% \note{address OpenSpiel's design of random node}
% \subsection{Episodic vs Continuous}
% \note{barely seen in the literature, need more literature review}
% \subsection{Self-Observability}
% % \note{agent needs to be able to observe itself}
% \subsection{Environment Output Structure}
% % The agent-environment interface specifies two return types from the environment, namely the \textbf{observation} and the \textbf{reward}.
% % All environment implementations used in the RL field follow 