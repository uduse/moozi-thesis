\section{Introduction} \label{sec:intro}

\textbf{Deep Learning (DL)} is a branch of \textbf{Artificial Intelligence (AI)} that emphasizes the use of neural networks to fit the inputs and outputs of a dataset.
The training of a neural network is done by computing the gradients of the loss function with respect to the weights and biases of the network.
A better trained neural network can better approximate the function that maps the inputs to the outputs of the dataset.

\textbf{Reinforcement Learning (RL)} is a branch of AI that emphasizes on solving problems through trials and errors with delayed rewards.
RL had most success in the domain of \textbf{game playing}, including boardgames and Atari games.
An extension to game playing is \textbf{General Game Playing (GGP)}, whose goal is to design a single algorithm that can play many different games without having much prior knowledge of the games.

\textbf{Deep Reinforcement Learning (DRL)} is a rising branch that combines DL and RL techniques to solve problems.
In a DRL system, RL techniques lay out the structure of the algorithm such as the use the \textbf{agent-environment interface} \note{reference}, a value function, a reward signal, e.t.c., while DL techniques are used approximate specific functions such as the value function.

\textbf{Planning} refers to any computational process that analyzes generated actions and their consequences in an environment.
In the RL terms, planning specifically means the use of a model to improve a policy.
RL algorithms that use planning had great success in playing boardgames, with the most significant achievement of beating human experts in Computer Go.

A \textbf{distributed system} is a computer system that uses multiple processes with various purposes to complete tasks.
DRL systems for solving large problems are both data and compute intensive.
Utilizing concurrency to increase efficiency and throughput for these DRL systems are sometimes necessary.
Building a distributed system to achieve such concurrency is a common practice in the industry but requires significant engineering effort.

\subsection{Contribution}
In this thesis we present the project \textbf{MooZi}, a general game playing system that play games by planning with a learned model.
The systems 
\begin{itemize}
    \item a collection of environment bridges that connect the system to various environments using a unified interface
    \item a neural network model that learns representation and could be used for planning
    \item a MCTS based planner that uses the learned model to perform planning.
    \item a distributed training system that efficiently trains the agent.
    \item a thesis with empirical studies and analysis
\end{itemize}
