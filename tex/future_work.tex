\chapter{Future Work}
% \moozi currently works well on 
\moozi is an on-going project and we aim to generalize \moozi even further and support more games.
We reported results of \moozi in \textit{Breakthrough}, but \moozi supports all other two-player perfect-information deterministic games in \textit{OpenSpiel}.
We will run \moozi in all such games and at the same time find a single set of hyperparameter that works well.
Currently \moozi assumes the environment dynamics to be deterministic.
A promising direction is to handle stochasticity and support non-deterministic video games and board games.
Prior work extended the \textit{MuZero} algorithm with encoded random nodes \cite{VectorQuantizedModels_Ozair.Li.ea_2021,PLANNINGSTOCHASTICENVIRONMENTS_Antonoglou.Schrittwieser.ea_2022}, and it will be our top priority to extend \moozi in a similar fashion.
\moozi currently does not support continuous action space and has poor support for large discrete action space.
We can extend \moozi to support continuous action space using \textit{Sampled MuZero} \cite{LearningPlanningComplex_Hubert.Schrittwieser.ea_2021}.
We can support large discrete action space better by reducing memory footprint through packing data with library that specializes in handling sparse N-dimensional arrays.
\moozi already supports planning with Gumbel \cite{PolicyImprovementPlanning_Danihelka.Guez.ea_2022} through MCTX.
We conducted preliminary experiments and we will follow up with a set of more rigorous experiments.
Moreover, \moozi currently interweaves handling episodes and timesteps.
Games such as \textit{Freeway} that has extremely long episodes require a large number of environment interactions per training step to generate at least one episode.
We will unify the handling of long episodes by slicing episodes into fix-sized sequences.
Moreover, experience represented as fix-sized sequences can be implemented using native JAX and speed up the system by making search targets (\(\mathbf{p}^{*}\) and $v^{*}$) stay on GPUs or TPUs in their lifetime.
In short, we will make \moozi more general as well as more efficient.