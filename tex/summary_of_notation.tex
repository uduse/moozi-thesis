\section{Summary of Notation}
% We adopt a similar notation to \citeauthor{ReinforcementLearningIntroduction_Sutton.Barto_2018} \cite{ReinforcementLearningIntroduction_Sutton.Barto_2018}.
% Capital letters are used for random variables, whereas lower case letters are used for the values of
% random variables and for scalar functions.
% Quantities that are required to be real-valued vectors are written in bold and in lower case (even if random variables).
% Matrices are bold capitals.

\begin{tabular}{| r | l |}
    \hline
    $s$ & state  \\
    $a$ & action \\
    $r$ & reward \\
    $t$ & timestep \\
    $T$ & terminal timestep \\
    $o$ & partially observable environment frame \\
    $\gamma$ & discount \\
    $x$ & hidden state \\
    $G^N$ & N-step return \\
    $\delta$ & TD-error \\
    $V$ & value function \\
    $Q$ & state-action value function \\
    $\mathcal{A}^e$ & environment action space \\
    $\mathcal{A}^a$ & agent action space, $ | \mathcal{A}^a | = | \mathcal{A}^e | + 1$ \\
    $\mathcal{S}$ & state space \\
    $\mathcal{O}$ & observation space \\
    $\mathcal{T}_t$ & step sample \\
    $\mathcal{T}$ & trajectory sample \\
    $\mathcal{L}$ & loss function \\

    \hline
    $B$ & batch size\\
    $H$ & height \\
    $W$ & width \\
    $C_e$ & environment channels \\
    $C_h$ & history channels \\
    $C_x$ & hidden space channels \\
    $K$ & number of unrolled steps \\
    $L$ & history length \\
    $N$ & bootstrap length for N-step return \\
    $A$ & dimension of action \\
    $Z$ & dimension of scalar transformation support \\
    \hline


\end{tabular}
